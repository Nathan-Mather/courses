%-----------------------------------------------------------------------------------
%	PACKAGES AND OTHER DOCUMENT CONFIGURATIONS
%----------------------------------------------------------------------------------



\documentclass[11pt]{article}

\usepackage[top=2cm, bottom=3cm, left=2cm, right=2cm]{geometry}

\setlength{\parindent}{0in}

\newcommand{\Var}{\mathrm{Var}}

\newcommand{\Cov}{\mathrm{Cov}}

\newcommand{\plim}{\rightarrow_{p}}

\usepackage{amsmath, amsfonts}
\usepackage{graphicx}
\usepackage{pdfpages}
\usepackage{bm}
\usepackage{listings}

% Expectation symbol
\newcommand{\E}{\mathrm{E}}
\newcommand{\V}{\mathrm{V}}
\newcommand{\N}{\mathcal{N}}

%----------------------------------------------------------------------------------
%	TITLE AND AUTHOR(S)
%----------------------------------------------------------------------------------

\title{Econ 675 Assignment 1} % The article title


\author{Nathan Mather} % The article author(s) 

\date{\today} % An optional date to appear under the author(s)

%----------------------------------------------------------------------------------
\begin{document}
	
%------------------------------------------------------------------------------
%	TABLE OF CONTENTS & LISTS OF FIGURES AND TABLES
%------------------------------------------------------------------------------
\maketitle % Print the title/author/date block

\setcounter{tocdepth}{2} % Set the depth of the table of contents to show sections and subsections only

\tableofcontents % Print the table of contents



%----------------------------------------------------
% Question 1
%---------------------------------------------------- 
\section{Question 1}

(2) and (3) differ because (2) is an elasticity holding utility constant. This means that as my wage changes I am compensated in some way so that my total utility is unchanged. However, it is possible to have the same total utility but have different marginal utility of wealth. For example, If my wage falls and I am compensated by more free time it is possible that my utility would remain the same, but my marginal utility of wealth is higher because I have less money. This would change the marginal benefit of working an additional hour and receiving the wage W. (3) holds the actual marginal utility of wealth constant as I change the wage, meaning that the marginal benefit of working another hour and receiving W more dollars has not changed. \\

Intuitively (3) would be bigger because Under (2) as my wage increases my wealth increases and lowers the marginal utility of an extra dollar. So I wont change my behavior as much as if that marginal utility per dollar was fixed . \\

In MaCurdy's model we have 
$$ \eta = \delta+\gamma(t) < \eta |_u = \delta + \gamma(t) -E(t) \theta < \eta|_{\lambda} = \delta
$$

Where $\delta$ is the intertemporal substitution elasticity, 





%------------------------------------------------
% end doc
%------------------------------------------------
\end{document}


