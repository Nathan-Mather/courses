%-----------------------------------------------------------------------------------
%	PACKAGES AND OTHER DOCUMENT CONFIGURATIONS
%----------------------------------------------------------------------------------



\documentclass[11pt]{article}

\usepackage[top=2cm, bottom=3cm, left=2cm, right=2cm]{geometry}

\setlength{\parindent}{0in}

\newcommand{\Var}{\mathrm{Var}}

\newcommand{\Cov}{\mathrm{Cov}}

\newcommand{\plim}{\rightarrow_{p}}

\usepackage{amsmath, amsfonts}
\usepackage{graphicx}
\usepackage{pdfpages}
\usepackage{bm}
\usepackage{listings}

% Expectation symbol
\newcommand{\E}{\mathrm{E}}
\newcommand{\V}{\mathrm{V}}
\newcommand{\N}{\mathcal{N}}

%----------------------------------------------------------------------------------
%	TITLE AND AUTHOR(S)
%----------------------------------------------------------------------------------

\title{Econ 675 Assignment 1} % The article title


\author{Nathan Mather} % The article author(s) 

\date{\today} % An optional date to appear under the author(s)

%----------------------------------------------------------------------------------
\begin{document}
	
%------------------------------------------------------------------------------
%	TABLE OF CONTENTS & LISTS OF FIGURES AND TABLES
%------------------------------------------------------------------------------
\maketitle % Print the title/author/date block

\setcounter{tocdepth}{2} % Set the depth of the table of contents to show sections and subsections only

\tableofcontents % Print the table of contents



%----------------------------------------------------
% Question 1
%---------------------------------------------------- 
\section{Question 1}

(2) and (3) differ because (2) is an elasticity holding utility constant. This means that as my wage changes I am compensated in some way so that my total utility is unchanged. However, it is possible to have the same total utility but have different marginal utility of wealth. For example, If my wage falls and I am compensated by more free time it is possible that my utility would remain the same, but my marginal utility of wealth is higher because I have less money. This would change the marginal benefit of working an additional hour and receiving the wage W. (3) holds the actual marginal utility of wealth constant as I change the wage, meaning that the marginal benefit of working another hour and receiving W more dollars has not changed. \\

Intuitively (3) would be bigger because Under (2) as my wage increases my wealth increases and lowers the marginal utility of an extra dollar. So I wont change my behavior as much as if that marginal utility per dollar was fixed . \\

In MaCurdy's model we have 
$$ \eta = \delta+\gamma(t) < \eta |_u = \delta + \gamma(t) -E(t) \theta < \eta|_{\lambda} = \delta
$$

Where $\delta$ is the intertemporal substitution elasticity, 



%----------------------------------------------------
% Question 2
%---------------------------------------------------- 
\section{Question 2}





%----------------------------------------------------
% Question 3
%---------------------------------------------------- 
\section{Question 3}
\subsection{a}
The estimate for $\delta$ is -.1254745. This is somewhat surprising as I would expect that as wages increase most people would work more and not less. Although, because of income effects it is possible to have a backwards bending labor supply curve. 
\\ \\ 

On the other hard there is division bias in the wage variable baising the result towards -1. Additionally wages are endogenous. The arrival of new information affects not only the wage rates, but also leads to a revision in expected lifetime wealth which also biases the result downwards. 

\subsection{b}
The estimate for $\delta$ is now -.004179. I am surprised it is still below zero but the confidence interval extends above zero. This moved in the expected direction. 
\subsection{c}
 They are jointly significant with an F test of  34.01. Historically, and in Macurdy's case, showing the F statistic is significant was enough. However this has been shown to be incorrect. Because there are many instruments, the requirement on the F statistic will be high. much higher than just joint significance. We can see that there will probably be an issue because many of these instruments are not significant on their own. 
 
 
 \subsection{d}
 under liml $\delta = .0094802 $. This is positive and higher than the estiamte in part b. I expected this as liml performs better under weak instroments than 2sls.
 
 \subsection{e}
 The Confidence interval based on Moreira's CLR test is $[ -.034,    .059]$ while the Anderson-Rubin condence interval is $ [-\infty, \infty] $. The later suggests that we have weak instruments. In fact, it suggests they are so weak we cannot make any inference. 
 
  \subsection{f}
  He does this because if we look beyond the basic F test, then we can see that many of the instruments are too weak to be used. For example, the AR confidence interval above tells us this. 
  
  \subsection{g}
  The joint f test for these 5 variables is  52.61. This is better than with all 13 and does not indicate weak instruments.
  
  \subsection{h}
  $\delta = .0628527 $ which is higher than with 14 instruments, as expected. Adding additional weak instruments only biases the result towards OLS which is biased downwards in this case. 
  
  \subsection{i}
  $\delta = .0726518 $ which is higher than the estimate in part d. Again, we expect this as we are using fewer weak instruments. 
  \subsection{j}
  The confidence interval is $ [ .01, .12]$
 %------------------------------------------------
% end doc
%------------------------------------------------
\end{document}


