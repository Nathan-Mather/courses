%-----------------------------------------------------------------------------------
%	PACKAGES AND OTHER DOCUMENT CONFIGURATIONS
%----------------------------------------------------------------------------------

\documentclass{beamer}


\newcommand{\Var}{\mathrm{Var}}

\newcommand{\Cov}{\mathrm{Cov}}

\newcommand{\plim}{\rightarrow_{p}}

\usepackage{apacite}

\usepackage{amsmath, amsfonts}
\usepackage{graphicx}
\usepackage{pdfpages}
\usepackage{bm}
\usepackage{listings}
\usepackage{multirow,array}
\usepackage{enumerate}
\usepackage{bbm}
\usepackage{subfig}
\usepackage{bbm}
\usepackage{multirow}



\usepackage{amssymb}

\usepackage{mathrsfs}
\usepackage{float}
\usepackage{booktabs}
\usepackage{color}
\usepackage{rotating}
\usepackage{amsthm}
\usepackage{multirow,array}
\usepackage{caption}
\usepackage{url}



\DeclareMathOperator*{\argmax}{arg\,max}
\DeclareMathOperator*{\argmin}{arg\,min}



% Expectation symbol
\newcommand{\E}{\mathrm{E}}
\newcommand{\V}{\mathrm{V}}
\newcommand{\N}{\mathcal{N}}
\newcommand{\R}{\mathbb{R}} 



%-----------------------------
% title stuff 
%-------------------------

\usepackage[utf8]{inputenc}


%Information to be included in the title page:
\title{Regressive Sin Taxes by Lockwood and Taubinsky: A Critical Review}
\author{Nathan Mather}
\institute{University of Michigan}
\date{2019}




%------------------------------------------------------------------------------
%	begin doc
%------------------------------------------------------------------------------

\begin{document}
	
	
	
	\frame{\titlepage}
	
\section{Introduction}
	
	\begin{frame}
	\frametitle{Introduction}

\textbf{Two conflicting forces at play when considering sin taxes }
\begin{itemize}
	\item People make bad decisions
	\begin{itemize}
		\item Correcting this can potentially increase welfare
		\item Similar logic to Pigouvian tax
	\end{itemize}
	\item Sin taxes can be regressive 
	\begin{itemize}
		\item Cigarettes and sugary drinks consumed disproportionately by the poor
		\item High efficiency subsidies disproportionately taken by rich
	\end{itemize}
\end{itemize}


\end{frame}


	\begin{frame}
\frametitle{Goal of Model}

\begin{itemize}

\item A model that addresses both of these concerns  \\~\\

\item Includes variable income tax. \\~\\

\item Consumers have heterogeneous earnings, abilities, and tastes, and can choose labor supply and consumption bundles\\~\\

\item Policy makers choose linear commodity tax and non-linear income tax. \\~\\

\item Policy maker and consumers disagree about what is best for them.

	
\end{itemize} 
\end{frame}




\begin{frame}
\frametitle{The Model}
\framesubtitle{The Environment}


\begin{itemize}
 \item Variable definitions


\begin{center}
\resizebox{.8\linewidth}{!}{\begin{tabular}{||c | c||} 
		\hline
		Variable & Meaning  \\ [0.5ex] 
		\hline\hline
		$\theta$ & Consumer Type \\ 
		\hline 
		$\mu(\theta)$ & Distribution of Type\\ 
		\hline
		$z$ & Earnings \\
		\hline
		$T(z)$ & nonlinear income tax \\
		\hline
		$c_2$ & Sin Good\\
		\hline
		$t$ &  Linear Commodit tax on $c_2$ \\ 
		\hline
		
		$c_1$ &  Numeraire good  \\ 
		\hline
		$p$ & Price of $c_2$ \\
		\hline
		$U(c_1, c_2.z;\theta)$ & Decision Utility\\
		\hline
		$V(c_1, c_2.z;\theta)$ & Policymaker ``correct" utility \\	\hline
		$\alpha(\theta)$ & Pareto Weights\\[1ex] 
		\hline
	\end{tabular}
}
\end{center}



	\item Functional form assumptions 
	\begin{itemize}
		\item $U$ is increasing and weakly concave in $c_1$ and $c_2$ and decreasing and strictly concave in $z$
	\end{itemize}
\end{itemize}

\end{frame}





\begin{frame}
\frametitle{The Model}
\framesubtitle{Policymaker's Problem}

\begin{itemize}
	\item Policymaker wants to maximize experienced utility $V$
	\begin{itemize}
		\item Weight consumers by Pareto weights 
		\item Can choose $T(\cdot)$ and $t$ to do this
		\item Subject to budget constraint
		\item Subject to individuals doing what they want
		
	\end{itemize}

$$  \max\limits_{T,t} \int \alpha(\theta) V(c_1(\theta), c_2(\theta), z(\theta); \theta)\mu(\theta)
$$

Subject to the budget constraint 

$$ \int (tc_2(\theta) + T(z(\theta)))\mu(\theta) = 0$$

and individual maximization 
$$ \{c_1(\theta),c_2\theta, z(\theta) \} = \argmax \limits_{c_1, c_2, z} U(c_1, c_2, z; \theta) $$
$$ s.t. \quad \ c_1 + (1+t)c_2 < z-T(z)\quad \forall\quad \theta$$
 
	
\end{itemize}
	
\end{frame}


\begin{frame}
\frametitle{The Model}
\framesubtitle{Difference between U and V}

\begin{itemize}
	\item Incorrect beliefs 
	\begin{itemize}
		\item Calorie content of food
		\item Health costs of food or drugs
		\item Energy efficiency of products
	\end{itemize}

	\item Limited attention or salience bias 
		\begin{itemize}
		\item People think "fat free" ice cream is healthy 
		\end{itemize}
	\item Present Bias/ Time Inconsistency
		\begin{itemize}
		\item Hyperbolic discounting ($\beta-\delta$ discounting)
		\item The model can treat $\beta$ as a bias.
		\item Policy maker could also weight present and future selves arbitrarily
		\end{itemize}

\end{itemize}

\end{frame}



\begin{frame}
\frametitle{The Model}
\framesubtitle{A Price Metric for Consumer Bias}



\begin{center}

\resizebox{1\linewidth}{!}{\begin{tabular}{||c | c||} 
		\hline
		Variable & Meaning  \\ [0.5ex] 
		\hline\hline
		$y $ & $z-T(z)$ \\ 
		\hline 
		$c_2(\theta, y, p, t, T)$ & Consumption chosen by individual of type $\theta$ given constraints  \\
		\hline
		$c_2^V(\theta, y, p, t, T)$  & What individual would choose if maximizing over V \\
		\hline
		$\gamma(\theta, z,t,T)$ or "Bias" & $\gamma$   s.t.   $c_2(\theta, y, p, t, T) = c_2^V(\theta , y - c_2\gamma, p-\gamma ,t, T)$ \\[1ex] 
		\hline
	\end{tabular} }
\end{center}

\begin{itemize}
	\item This is the compensated price change that produces the same effect on demand as the bias does
	\item In some cases this can be measure directly 
	\begin{itemize}
		\item Chetty et al. (2009)
		\begin{itemize}
			\item  Tax salience
			\item $\Delta$ price that alters demand as much as tax-inclusive price
		\end{itemize}  
	\end{itemize}
\end{itemize}


\end{frame}










\begin{frame}
\frametitle{The Model}
\framesubtitle{Redistributive Motives}
\begin{itemize}
	\item Marginal Social welfare weights 
	\begin{itemize}
		\item Marginal social welfare generated by a marginal unit of consumption of $c_1$ for a given individual 
		\item Formally, $g(\theta) = \alpha(\theta)V_1/\lambda$
		\item $\bar{g} = \int_{\Theta}^{}g(\theta)d\mu(\theta)$
		\item If there are no income effects on consumption and labor
		supply, then $\bar{g} =1$ by construction.
	\end{itemize}
\item Formulas for optimal taxes will thus depend on the policymaker’s (or society’s) preferences for wealth equality
	
\end{itemize}

\end{frame}






\begin{frame}
\frametitle{Optimal Tax With Discrete Types}

\begin{itemize}
	\item $\theta \in {L,H}$
	\item $w_L < w_H$
	\item Internality  is harmful $\gamma(\theta) > 0$
	\item $L$ consumes more $c_2$ than $H$
	\item Normalize $c_2$ so $p=1$
	\item $c_1^*(\theta) = z^*(\theta) - T_{\theta} - (1+t)c_2^*(\theta)$
	
\end{itemize}

\end{frame}






\begin{frame}
\frametitle{Eample 1: Regressivity Caused by Heterogeneous Preferences   }

\textbf{Functional Form Assumptions}

$$ U(c_1, c_2,z;\theta) = G(c_1 + v(c_2, \theta) - \Psi(z/w_{\theta}) $$
$$ V(c_1, c_2,z;\theta) = G(c_1 + v(c_2, \theta) - \gamma(\theta)c_2 - \Psi(z/w_{\theta})
$$
\begin{itemize}
	\item $c_2^*(H) < c_2^*(L)$
	\item G is concave
	\item No income effects for choice of $c_2$ or labor supply 
\end{itemize}
\end{frame}
	
	
	
	
\begin{frame}
\frametitle{Eample 1: Regressivity Caused by Heterogeneous Preferences }
\framesubtitle{Policy Maker's problem}

Policymaker solves 
$$\max\limits_{t, T_L, T_H} \sum_{\theta}^{} V(c_1^*(\theta), c_2^*(\theta, z^*(\theta);\theta)\mu(\theta)  $$

S.T. 
$$ \frac{1}{2} \sum_{\theta}^{}(T_{\theta} + tc_2^*(\theta)) \geq 0 $$

and 

$$ (c_1^*(\theta), c_2^*(\theta), z^*(\theta)) \quad \text{Maximizes } U(c_1,c_2,z;\theta) \quad \text{given cnstraints}$$

\end{frame}

\begin{frame}
\frametitle{Eample 1: Regressivity Caused by Heterogeneous Preferences }
\framesubtitle{result}
$$
t^* = \underbrace{ \frac{\sum_{\theta}^{} g(\theta)\gamma(\theta)\frac{dc_2^*(\theta}{dt}}{\sum_{\theta}\frac{dc_2^*(\theta)}{dt}}}_{ \text{corrective benefits}}
-
\underbrace{\frac{\sum_{\theta}c_2^*(\theta)(g(\theta) -1)}{\sum_{\theta}\frac{dc_2^*(\theta)}{dt}}}_{\text{Regressivity Costs}}
$$

\begin{itemize}


	\item  *NOTE: in the paper they incorrectly have $1-g(\theta)$ in the second term \\~\\

	\item  Correction is more valuable with greater bias $\gamma(\theta)$ and higher welfare weight $g(\theta)$ \\~\\ 

	\item  Regressivity cost reduces optimal tax since $g(L) >1$, $g(H) < 1 $, and $c_2^*(H) < c_2^*(L)$ 

\end{itemize}


\end{frame}




\begin{frame}
\frametitle{Example 2: Regressivity Caused by Income Effects }

\textbf{Functional Form Assumptions}
$$ U(c_1, c_2,z;\theta) = G(c_1 + v(c_2, c_1) - \Psi(z/w_{\theta}) $$
$$ V(c_1, c_2,z;\theta) = G(c_1 + v(c_2, c_1) - \gamma(\theta)c_2 - \Psi(z/w_{\theta})
$$

\textbf{Perturbation argument}
Raise commodity tax and adjust income tax to neutralize effect on wealth. At the optimum, this has zero first oder effect on welfare. Giving 
$$ \underbrace{ t\left( \sum_{\theta} \frac{dc_2^*(\theta)}{dt} \biggr\rvert_{u} \right)}_{ \text{Effect on Gvernment Revenue}} - 
\underbrace{ \sum_{\theta} \left( g(\theta) \gamma \frac{dc_2^*(\theta)}{dt} \biggr\rvert_{u} \right)}_{ \text{Effect on Consumer Welfare}} =0
$$ 


\end{frame}

\begin{frame}
\frametitle{Example 2: Regressivity Caused by Income Effects }

\textbf{Result}
$$
t^* =  \frac{\sum_{\theta}^{} \left( g(\theta)\gamma\frac{dc_2^*(\theta)}{dt} \biggr\rvert_{u} \right)}{\sum_{\theta}\frac{dc_2^*(\theta}{dt} \biggr\rvert_{u}}
$$

\begin{itemize}

\item No Regressivity costs in this case
\item  Income tax reform can perfectly neutralize the effects of the commodity tax on income

\end{itemize}
\end{frame}

\begin{frame}
\frametitle{Understanding The Difference}

\begin{itemize}
	\item Progressive taxes make people work less
	\item Heterogeneous Preferences 
	\begin{itemize}
		\item Changing income will not alter consumption 
		\item $c_2$ tax is regressive from societal standpoint 
		\item Not regressive for individual.
		\begin{itemize}
			\item Doesn't alter $z$
		\end{itemize} 
		\item Progressive income tax lowers $z$
	\end{itemize}
	\item Income Effects 
	\begin{itemize}
		\item $c_2$ good is inferior
		\item $c_2$ tax is regressive from societal standpoint 
		\item $c_2$ tax is also regressive for individual
		\begin{itemize}
			\item If I work more, I can buy less $c_2$ and avoid the tax 
			\item leads to higher $z$
		\end{itemize}
	
		\item Progressive income tax lowers $z$
		\begin{itemize}
			\item $z$ effects offset, total output unchanged 
		\end{itemize}
	\end{itemize}

\end{itemize}


\end{frame}




\begin{frame}
\frametitle{A General Formula for The Optimal Commodity Tax}
\framesubtitle{Assumptions and elasticity concepts}

\textbf{assumptions}
\begin{itemize}
	\item No Labor supply mis-optimization
	\item Constant Marginal Social Welfare weights conditional on income 
	\item U and V are smooth, strictly concave in $c_1, c_2, z$ and $\mu$ is differentiable with full support 
	\item $T(\cdot)$ is twice differentiable and each consumer's choice of income $z$ admits a unique global optimum
	
\end{itemize}

\end{frame}




\begin{frame}
\frametitle{A General Formula for The Optimal Commodity Tax}
\framesubtitle{Assumptions and elasticity concepts}

\textbf{Parameters}
\begin{itemize}
	\item $\zeta(\theta,t,T)$: Price elasticity of demand for $c_2$ of type $\theta$
	\item  $\zeta^c(\theta,t,T)$: Compensated price elasticity of demand for $c_2$
	\item $\eta(\theta, t,T)$: The income effect on $c_2$ Equal to $\zeta -\zeta^c $
	\item  $\zeta^c_z(\theta,t,T)$: The compensated elasticity of taxable income with respect to the marginal income tax rate 
	\item $\eta_z(\theta, t,T)$: Income effect on labor supply
	
\end{itemize}

\end{frame}






\begin{frame}
\frametitle{A General Formula for The Optimal Commodity Tax}
\framesubtitle{Assumptions and elasticity concepts}


\begin{itemize}
	\item $\bar{X}(z)$ is the average of Variable X for given income z 
	\item $C_2$ is $\int_{\Theta}^{}1{z(\theta ) \leq z}d\mu(\theta)$
	\item 	$H(z)$ is the income Distribution 
	\item $\phi(z)$ is how much $c_2$ an average z-earner would consume if all variation in $c_2$ was explained solely by income effects. 
	\item Let $\tilde{\phi}(z) := \frac{\bar{c}_2(z) - \phi(z)}{C_2}$ \\~\\ 
	
	\begin{itemize}
		\item This measures how much difference between $\bar{c}_2(z)$ and $\bar{c}_2(0)$ is explained by preference heterogeneity. (normalize by average $c_2$)
	\end{itemize}
	
\end{itemize}

\end{frame}



\begin{frame}
\frametitle{A General Formula for The Optimal Commodity Tax}
\framesubtitle{An expression for the optimal commodity tax 1}

\textbf{Average Marginal Bias}
$$ \bar{\gamma}(t,T) = \frac{\int_{\Theta}\gamma(\theta,t,T) \left( \frac{dc_2(\theta,t,T)}{dt}  \biggr\rvert_{u} \right)d\mu(\theta)}{\int_{\Theta}\left( \frac{dc_2(\theta,t,T)}{dt}  \biggr\rvert_{u} \right)d\mu(\theta)}$$

\textbf{Average Marginal Bias Given z}
$$ \bar{\gamma}(z,t,T) = \frac{\int_{\Theta}\gamma(\theta,t,T) \left( \frac{dc_2(\theta,t,T)}{dt}  \biggr\rvert_{u} \right) 1 \{ z(\theta,t,T) = z \} d \mu(\theta)  }    {\int_{\Theta}\left( \frac{dc_2(\theta,t,T)}{dt}  \biggr\rvert_{u} \right) 1 \{ z(\theta,t,T) = z \} d\mu(\theta)}$$


This is the marginal bias weighted by individuals marginal responses to a compensated change in t.


\end{frame}

\begin{frame}
\frametitle{A General Formula for The Optimal Commodity Tax}
\framesubtitle{An expression for the optimal commodity tax 1}

\textbf{ Covariance of welfare weight with consumption-weighted bias and elasticity}

$$ \sigma := \Cov_H \left[ g(z), \frac{\bar{\gamma(z)}}{\bar{\gamma}} \frac{\bar{\zeta^c}(z)}{\bar{\zeta^c}}
\frac{\bar{c}_2(z)}{C_2 }  \right] $$

This captures the extent to which bias correction is concentrated on the low-end of the income distribution

\end{frame}



\begin{frame}
\frametitle{A General Formula for The Optimal Commodity Tax}
\framesubtitle{An expression for the optimal commodity tax 1}

\begin{itemize}
	\item Start by using social marginal utility of income  $\hat{g}(z)$ rather than social marginal welfare weights.
	\item Average welfare effect of marginally increasing the incomes of consumers currently earning income
	z.
	\item rather than marginally increasing numeraire consumption $c_1$
	\item This accounts for fiscal externalities resulting from income effects, and for the fact that some of this additional consumption will be mis-spent due to bias.
\end{itemize} 

\end{frame}



\begin{frame}
\frametitle{A General Formula for The Optimal Commodity Tax}
\framesubtitle{An expression for the optimal commodity tax 1}

\textbf{Proposition 1} 

\begin{align}
t &= \bar{\gamma}(\bar{g} + \sigma) - \frac{p+t}{\bar{\zeta}^c} \Cov \left[\hat{g}(z), \tilde{\phi}(z) \right] \\
&= \frac{\bar{\zeta}^c \bar{\gamma} (\bar{g}  + \sigma) - p\Cov \left[ \hat{g}(z),\tilde{\phi}(z) \right]}{\bar{\zeta}^c + \Cov \left[\hat{g}(z), \tilde{\phi}(z) \right]}
\end{align}


\begin{itemize}

\item Corrective benefit is increase in 
	\begin{itemize}
	 \item Average marginal bias $\bar{\gamma}$
	 \item  Average social welfare weight $\bar{g}$
	 \item Extent to which bias correction is concentrated with low income consumers $\sigma$
	\end{itemize}
\item $\Cov \left[ \hat{g}(z), \tilde{\phi}(z) \right]$ is roughly regressivity cost that cannot be offset by progressive income taxes. 
\begin{itemize}
	\item Depends on extent to which $c_2$ differential is due to preference heterogeneity or income effects. 
\end{itemize}
\end{itemize}

\end{frame}




\begin{frame}
\frametitle{A General Formula for The Optimal Commodity Tax}
\framesubtitle{An expression for the optimal commodity tax 2}

\textbf{Lemma 2} Let $\chi(z) := \phi(z) - \int_{0}^{z} w(x,z) \frac{\eta_z}{\zeta_z^c x}(c_2(x) - \phi(x))dx$, where $w(x,z) = e^{\int_{z'=x}^{x'=z}\frac{\eta_z}{\zeta_z^c z}}dx'$. Then increasing the commodity tax by dt and decreasing the income tax by $\chi(z)dt$ leaves the average labor supply of z-earners unchanged. \\~\\ 

$\chi(z) := \phi(z)$ when $\eta_z = 0$. i.e. when there are no labor supply income effects. \\~\\ 

Define $\tilde{\chi}(z) := \frac{\bar{c}_2(z) - \chi(z)}{C_2}$
 
\end{frame}


\begin{frame}[shrink=6]
\frametitle{A General Formula for The Optimal Commodity Tax}
\framesubtitle{An expression for the optimal commodity tax 2}

\textbf{Proposition 2} The optimal commodity tax t satisfies. 

$$ t = \underbrace{ \bar{\gamma}(\bar{g} + \sigma)}_\text{corrective benefits} + 
\underbrace{ \frac{p + t}{\bar{\zeta}^c} \E[(g(z)-1) \tilde{\chi}(z)] }_\text{regressivity costs}  - 
\underbrace{ \frac{1}{\bar{\zeta}^c} \int \tilde{\chi}(z) \eta (z) (t-g(z) \bar{\gamma}(z))}_\text{additional impact from income effect}$$

\textbf{In the absence of income effects} 

$$ t = \bar{\gamma}(\bar{g} + \sigma) -  \frac{p + t}{\bar{\zeta}^c} \Cov \left[ g(z), \tilde{\phi}(z)  \right] $$
	 
\end{frame}







\begin{frame}
\frametitle{Interpretations and Implication}
\framesubtitle{Optimal taxes in the Absence of Redistributive Concerns}

\textbf{Corollary 2} suppose that either\\
 1) $z(\theta)$ is constant in $\theta$ or\\
  2) $g(\theta) = 1$ $\forall$ $\theta$ \\
  Then $t^* = \bar{\gamma}$ (From Proposition 1). \\~\\

\begin{itemize}
\item Optimal commodity tax exactly offsets the average marginal bias. 
\end{itemize}

\end{frame}



\begin{frame}
\frametitle{Interpretations and Implication}
\framesubtitle{Optimal taxes in the Absence of corrective Concerns}

\textbf{When there are no corrective concerns}
$$ t = -\frac{p\Cov \left[ \hat{g}(z),\tilde{\phi}(z) \right]}{\bar{\zeta}^c + \Cov \left[\hat{g}(z), \tilde{\phi}(z) \right]} $$ 

\begin{itemize}
\item The Atkinson-Stiglitz theorem itself obtains as a special case of (6) when all variation
in $c_2$ consumption is driven by income effects, which then implies that t = 0
\end{itemize}
\end{frame}



\begin{frame}
\frametitle{Interpretations and Implication}
\framesubtitle{ Optimal Taxes When Income Effects do not Affect $c_2$ consumption}

\textbf{Corollary 3} Suppose that there are no income effects: $\eta \equiv 0$ and $\eta_z \equiv 0 $ then 

$$ t = \underbrace{\bar{\gamma}(\bar{g} + \sigma)}_\text{Corrective Benefits} -  \underbrace{\frac{p + t}{\bar{\zeta}^c} \Cov \left[ g(z), \tilde{\phi}(z)  \right]}_\text{Regressivity Costs} $$

\begin{itemize}
	\item Generalizes the result in Example 1
	\item First term now depends of $\sigma$ (concentration of corrective benefits among low income)
	\item Second term persists because progressive income tax. Fiscal externalities outweigh re-distributive benefit 
	\item As consumption of $c_2$ becomes inelastic, t become a sin subsidy. 
\end{itemize}
\end{frame}






\begin{frame}
\frametitle{Interpretations and Implication}
\framesubtitle{Optimal taxes when all differences in $c_2$ consumption are due to income effects }

\textbf{Corolalry 4} Suppose that $U_2(c_1,c_2, \theta, z) / U_1(c_1,c_2, \theta, z)$ is constant in $\theta$ for each $z$. Then 

$$ t^* = \bar{\gamma}(\bar{g} + \sigma)$$

\begin{itemize}
	\item Generalizes Example 2 
	\item higher $\sigma$ implies higher benefit to bias correction
	\item Policymaker will spend more that \$1 to eliminate \$1 mistake made by poor consumers. 
\end{itemize}
\end{frame}


\begin{frame}
\frametitle{Interpretations and Implication}
\framesubtitle{The key role of the price elasticity of demand in determining the importance of corrective benefits}

\textbf{Recall From Proposition 1} 

$$ t = \frac{\bar{\zeta}^c \bar{\gamma} (\bar{g}  + \sigma) - p\Cov \left[ \hat{g}(z),\tilde{\phi}(z) \right]}{\bar{\zeta}^c + \Cov \left[\hat{g}(z), \tilde{\phi}(z) \right]}$$ 

\begin{itemize}
	\item As elasticity grows large, corrective benefits per unit of tax grow large
	\item As elasticity gets small, corrective benefits become negligible
	\item Elasticity low enough can imply a subsidy on sin goods.
\end{itemize}


\end{frame}




\begin{frame}
\frametitle{extensions} 
\begin{itemize}
	\item Tax salience on the labor supply margin 
	\begin{itemize}
		\item Effect of commodity taxes on labor supply my be minimal 
		\item If people don't consider commodity taxes in in labor supply, moves us closer to preference heterogeneity case.
	\end{itemize}
	\item N $>$ 2 Dimension of Consumption 
	\begin{itemize}
		\item Considers substitutability of goods
	\end{itemize}
	\item Externalities 
	\begin{itemize}
		\item Special case of this framework 
	\end{itemize}
	\item Without te First-order approach
	\item labor supply misoptimization 
\end{itemize}

\end{frame}

\begin{frame}
\frametitle{Conclusion} 

\begin{itemize}
	\item reconciles the role for corrective taxes with the concern that such taxes may be regressive
	\item Clarifies that the optimal policy depends on a number of statistics. 
	\begin{itemize}
		\item Preference heterogeneity vs. Income effects 
		\item Bias of both rich and poor 
		\item Elasticity of demand and how it varies across income
		\item salience of commodity taxes on labor supply margin 
	\end{itemize}

\end{itemize}

\end{frame}
	
	
	\begin{frame}
	\frametitle{Citation} 
	
	B. Lockwood and D. Taubinsky, “Regressive Sin Taxes,” NBER WP No. 23085, March 2017.
	
	\end{frame}
	
	
%------------------------------------------------
% end doc
%------------------------------------------------
\end{document}
	