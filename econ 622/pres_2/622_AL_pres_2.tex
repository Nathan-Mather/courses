%-----------------------------------------------------------------------------------
%	PACKAGES AND OTHER DOCUMENT CONFIGURATIONS
%----------------------------------------------------------------------------------

\documentclass{beamer}


\newcommand{\Var}{\mathrm{Var}}

\newcommand{\Cov}{\mathrm{Cov}}

\newcommand{\plim}{\rightarrow_{p}}

\usepackage{apacite}

\usepackage{amsmath, amsfonts}
\usepackage{graphicx}
\usepackage{pdfpages}
\usepackage{bm}
\usepackage{listings}
\usepackage{multirow,array}
\usepackage{enumerate}
\usepackage{bbm}
\usepackage{subfig}
\usepackage{bbm}
\usepackage{multirow}



\usepackage{amssymb}

\usepackage{mathrsfs}
\usepackage{float}
\usepackage{booktabs}
\usepackage{color}
\usepackage{rotating}
\usepackage{amsthm}
\usepackage{multirow,array}
\usepackage{caption}
\usepackage{url}



\DeclareMathOperator*{\argmax}{arg\,max}
\DeclareMathOperator*{\argmin}{arg\,min}



% Expectation symbol
\newcommand{\E}{\mathrm{E}}
\newcommand{\V}{\mathrm{V}}
\newcommand{\N}{\mathcal{N}}
\newcommand{\R}{\mathbb{R}} 

\setbeamertemplate{navigation symbols}{}
 
\addtobeamertemplate{navigation symbols}{}{%
	\usebeamerfont{footline}%
	\usebeamercolor[blue]{footline}%
	\hspace{1em}%
	\insertframenumber/\inserttotalframenumber
}



%-----------------------------
% title stuff 
%-------------------------

\usepackage[utf8]{inputenc}



%Information to be included in the title page:
\title{Asymmetric Learning Model With Wage Rigidity and Costly Firings}
\author{Nathan Mather}
\institute{University of Michigan}
\date{2019}





%------------------------------------------------------------------------------
%	begin doc
%------------------------------------------------------------------------------

\begin{document}
	

	
	\frame{\titlepage}
	
\section{Introduction}

%------------------------------------------------------------------------------
%	frame 
%------------------------------------------------------------------------------
\begin{frame}
	\frametitle{Introduction}
	\begin{itemize}
		\setlength{\itemsep}{3mm}
		\item Analyze an asymmetric learning model as it relates to wages and tenure 
		\item Incorporate sticky wages 
		\item Incorporate fixed firing costs 
		\item Basic model set up is similar to Acemoglu and Pischke
		\item Scaled back to focus on wage evolution 
		\item Simplifying in certain ways allows adding other unexplored complexity 
	\end{itemize}
\end{frame}


%------------------------------------------------------------------------------
%	frame 
%------------------------------------------------------------------------------
\begin{frame}
\frametitle{Outline}
\begin{itemize}
	\setlength{\itemsep}{3mm}
	\item Outline of basic model structure 
	\item Equilibrium under flexible wages 
	\item Equilibrium under sticky wages
	\item Equilibrium under sticky wages and various firing costs
	\item Implications 
	\item Thoughts moving forward 
\end{itemize}
\end{frame}

%------------------------------------------------------------------------------
%	frame 
%------------------------------------------------------------------------------
\begin{frame}
\frametitle{Model Environment}
\framesubtitle{My Model}
\begin{itemize}
	\setlength{\itemsep}{1mm}
	\item Workers have ability $\theta \in [0,1]$ 
	\begin{itemize}
		\setlength{\itemsep}{1mm}
		\item Produce $\theta$ per period 		
	\end{itemize}
	\item Workers supply labor inelastically
	\item exogenously separate at rate $\delta$ 
	\item Employers receive a good signal $g$ w.p. $\theta$ and a bad signal $b$ w.p. $1-\theta$
	\item Employers offer wages each period (no long term contracts).
	\begin{itemize}
		\item All firms can condition on employment status 
		\item Current employer can also condition wage on signal
	\end{itemize}
	\item Outside firms see employment status and history (resume)
	\item There are many identical firms
	\begin{itemize}
		\item Zero profit condition
	\end{itemize} 
	\item Downward wage rigidity 
	\item Fixed cost to fire employee
	
\end{itemize}
\end{frame}



%------------------------------------------------------------------------------
%	frame 
%------------------------------------------------------------------------------
\begin{frame}
\frametitle{Variable Definitions}

\begin{center}
	\resizebox{.75\linewidth}{!}{\begin{tabular}{||c | c||} 
			\hline
			Variable & Meaning  \\ [0.5ex] 
			\hline\hline
			$\theta$ & Ability \\ 
			\hline 
			$g$ & Good Signal\\ 
			\hline
			$b$ & Bad Signal \\
			\hline
			$e$ & Employed Signal\\
			\hline
			$f$ & Fired Signal\\
			\hline
			$F_C$ &  Fixed Cost to Firing \\ 
			\hline
			$\delta$ & Exogenous separation rate \\ 
			\hline
			$w_1$ &  Period 1 Wage \\ 
			\hline
			$w_g$ & Wage After Good Signal \\
			\hline
			$w_b$ & Wage After Bad Signal \\
			\hline
			$w_u$ & Wage for unemployed worker\\	\hline
			$\pi$ & Profits\\[1ex] 
			\hline
		\end{tabular}
	}
\end{center}

\end{frame}


%------------------------------------------------------------------------------
%	frame 
%------------------------------------------------------------------------------
\begin{frame}
\frametitle{Time Line}


	\begin{itemize}
			\setlength{\itemsep}{3mm}
		\item period 1 wage offers
		\item Workers produce output
		\item workers exogenously separate
		\item Workers send signal of ability 
		\item Employers decide who to fire
		\item Employers offer period 2 wages conditional on signals 
		\item Outside Firms offer wages conditional on resume 
		\item Workers take the best offer and work for one more period 
		\item Workers retire 
	\end{itemize}

\end{frame}





%------------------------------------------------------------------------------
%	frame 
%------------------------------------------------------------------------------
\begin{frame}
\frametitle{Flexible Wages}
\framesubtitle{}

\begin{itemize}
	\setlength{\itemsep}{3mm}
	\item Let $Q$ denote the fraction of workers that don't separate but then quit.  
	
	$$ w_u(Q) = (\delta \E[\theta] + Q \E[\theta | \text{worker quit}])/(\delta + Q)$$
	
	\item Employers will not pay more than the expected output of a worker, giving
	
	$$ w_b = \E[\theta | b] $$
	
	\item This implies that all bad signal employees quit since if $\delta > 0$ 
	
	$$ \E[\theta|b] < \frac{ \delta \E[\theta] + (1-\delta)p(b)( \E[\theta | b])}{\delta + (1-\delta)p(b)} = \min w_u(Q) $$
	
	\item Where the Right hand side is the lowest possible $w_u$ since here $ Q = (1-\delta)p(b)$
\end{itemize}
\end{frame}


%------------------------------------------------------------------------------
%	frame 
%------------------------------------------------------------------------------
\begin{frame}
\frametitle{Flexible Wages}
\framesubtitle{}

\begin{itemize}
	\setlength{\itemsep}{3mm}
	\item If all bad signal employees quit than anyone still employed is identified as high ability. Giving
	
	$$ w_g = \E[\theta|g]$$
	
	\item This gives 
	
	$$ w_u =  \frac{ \delta \E[\theta] + (1-\delta)p(b)( \E[\theta | b])}{\delta + (1-\delta)p(b)}$$
	
	\item And a first period wage 
	
	$$w_1 = \E[\theta]$$

\end{itemize}
\end{frame}


%------------------------------------------------------------------------------
%	frame 
%------------------------------------------------------------------------------
\begin{frame}
\frametitle{Sticky Wages}

\begin{itemize}
	\setlength{\itemsep}{5mm}
	\item Now employers can't cut their low signal worker's wages. 
	\item We know $w_1 > \E[\theta|b]$ because if it weren't profits would be positive
	\item This implies employers fire bad signal workers 
	\item Wage outcomes are identical to flexible model but with workers fired rather than quitting 
	
\end{itemize}

\end{frame}


%------------------------------------------------------------------------------
%	frame 
%------------------------------------------------------------------------------
\begin{frame}
\frametitle{Sticky Wages and fixed firing costs}

\framesubtitle{Low firing costs}

\begin{itemize}
	\setlength{\itemsep}{3mm}
	\item A low firing cost will not alter firms decision to fire bad signal employees 
	\item  Now that hiring an employee means I may have to pay a cost to fire them, their marginal benefit is decreased and so their wage offer is lower
	
	$$ w_1 = \E[\theta] + (1- \delta)p(b)F_C$$
	
	\item Firms continue to do this as long as
	
	$$ w_1 - \E[\theta|b] > F_C$$
	
$$ \implies  \E[\theta] + (1- \delta)p(b)F_C -  \E[\theta|b] > F_C $$

$$ \implies f_C < \frac{\E[\theta] -  \E[\theta|b]}{1 + p(b)(1-\delta)} $$
	
\end{itemize}

\end{frame}



%------------------------------------------------------------------------------
%	frame 
%------------------------------------------------------------------------------
\begin{frame}
\frametitle{Clarifying Equilibrium wages}
\framesubtitle{An Aside}

\textbf{To clarify the wage offers in the next section, first consider this numerical example: }

\begin{itemize}
	\item $w_u = 1 $
\end{itemize}

\begin{center}
	\begin{tabular}{||c | c||} 
		\hline
		$\E[\theta]$ & EQ Wage offer  \\ [0.5ex] 
		\hline\hline
		3 & 2.5 \\ 
		\hline 
		2 & 2\\[1ex] 
		\hline
	\end{tabular}
	
\end{center}

\begin{itemize}
	
	\item If the "raiding" firm offers $w = 2 + \epsilon$ to all  employed workers, they get only those with $E[\theta] = 2$ for a loss 
	
	\item If they offer $ w = 2.5 + \epsilon$ they get workers with average $\E[\theta] = 2.5$ for a loss
	
	\item These are the lowest wages the firm can offer and still retain their employees
	
	\item In general, it means employers pay a worker the expected ability of all workers less than and equal to their own ability
	
\end{itemize}
\end{frame}




%------------------------------------------------------------------------------
%	frame 
%------------------------------------------------------------------------------
\begin{frame}
\frametitle{Sticky Wages and fixed firing costs}

\framesubtitle{High firing costs}

\begin{itemize}
	\setlength{\itemsep}{3mm}
	\item If the fixed costs become sufficiently large, it is clear that firms will not fire any employees
	
	\item Firms must pay their employees enough to keep them from getting "raided"
	
	$$ w_g = E[\theta | e] =\E[\theta]$$
	
	$$ w_b = w_1$$ 
	
	\item Unemployed workers get their expected output 
	
	$$ w_u = E[\theta] $$
	
\end{itemize}

\end{frame}

%------------------------------------------------------------------------------
%	frame 
%------------------------------------------------------------------------------
\begin{frame}
\frametitle{Sticky Wages and fixed firing costs}

\framesubtitle{High firing costs}

\begin{itemize}
	\setlength{\itemsep}{3mm}
	\item This gives a first period wage through the zero profit condition 
	$$w_1 = \E[\theta] + (1-\delta)(p(g)(\E[\theta|g] - \E[\theta]) + p(b)(\E[\theta|b] - w_1))$$
	\item using the following fact: $p(g)\E[\theta|g] + p(b)\E[\theta|b] = \E[\theta]$
	$$ \implies w_1 = \E[\theta] +  (1-\delta)(\E[\theta] - p(g)\E[\theta] - p(b)w_1)$$
	\item Next, using 
	$$ \E[\theta] - p(g)\E[\theta] = p(b)\E[\theta]$$
	$$ \implies  w_1 = \E[\theta] +  (1-\delta)p(b)(\E[\theta] - w_1) $$
	$$ \implies w_1 +(1-\delta)p(b)w_1 = \E[\theta] + (1-\delta)p(b)\E[\theta]$$
	$$ \implies w_1 = \E[\theta]$$
	\item all workers get paid $\E[\theta]$
	
	
\end{itemize}

\end{frame}

%------------------------------------------------------------------------------
%	frame 
%------------------------------------------------------------------------------


\begin{frame}
\frametitle{Sticky Wages and fixed firing costs}

\framesubtitle{High firing costs: stability}


\begin{large}
	\textbf{This equilibrium will be stable whenever it is more costly to fire an employee then to keep them at this wage.}
\end{large}


$$ F_C >  w_1 - \E[\theta|b] = \E[\theta] - \E[\theta|b]$$



\end{frame}


%------------------------------------------------------------------------------
%	frame 
%------------------------------------------------------------------------------

\begin{frame}

\frametitle{Sticky Wages and fixed firing costs}

\framesubtitle{Mixed Equilibrium}

\begin{itemize}
	\setlength{\itemsep}{3mm}
	\item Now consider the case between the two extremes.
	
	$$ F_C \in \left[  \frac{\E[\theta] - \E[\theta|b]}{1 + p(b)(1-\delta)} , \E[\theta] - \E[\theta|b] \right] $$
	
	\item In this case employers will fire a fraction of their bad signal workers $\delta_F$ until they are indifferent between firing and keeping them 
	
	$$  F_C = w_1 - E[\theta|b]$$ 
	
	\item Good signal employees need to be paid the expected output of an average worker employee in period 2 
	
	$$ w_g(\delta_F) = \frac{p(g)\E[\theta|g] + p(b)(1-\delta_F) \E[\theta|b] }{p(g) + p(b)(1-\delta_F)} $$
	

\end{itemize}
\end{frame}

%------------------------------------------------------------------------------
%	frame 
%------------------------------------------------------------------------------


\begin{frame}

\frametitle{Sticky Wages and fixed firing costs}

\framesubtitle{Mixed Equilibrium}

\begin{itemize}
	\setlength{\itemsep}{3mm}

	\item bad signal employees are paid the lowest wage possible $ w_b = w_1$ and unemployed get their expected output 
	
	$$ w_u = \frac{ \delta\E[\theta] + (1-\delta)\delta_F\E[\theta | b]}{  \delta + (1-\delta)\delta_F} $$
	
	\item by the zero profit condition we get the first period wages 
	
	$$w_1 = \E[\theta] + (1-\delta)  \bigg( p(g) \Big(\E[\theta|g] - w_g(\delta_F)  \Big) - p(b)F_C  \bigg) $$
	
\end{itemize}
\end{frame}


%------------------------------------------------------------------------------
%	frame 
%------------------------------------------------------------------------------


\begin{frame}

\frametitle{Sticky Wages and fixed firing costs}

\framesubtitle{Mixed Equilibrium}

\begin{itemize}
	\setlength{\itemsep}{3mm}
	
	\item We have two equations for $w_1$ with the only unknown being $\delta_F$
	
	\item I have solved for it, but it not very intuitive or generally helpful. Can use it to solve numeric examples.
	
	\newcommand{\TA}{\frac{\E[\theta] - F_C - (1-\delta)p(b)F_C - \E[\theta|b]}{(1-\delta)p(g)} + \E[\theta|g]}
	
	\newcommand{\TAMEG}{\frac{\E[\theta] - F_C - (1-\delta)p(b)F_C - \E[\theta|b]}{(1-\delta)p(g)}}
	
	$$ \delta_F = \frac{p(g) \left[\TAMEG \right]}{ p(b) \left[ \E[\theta | b] - \TA \right] } $$
	
\end{itemize}
\end{frame}

%------------------------------------------------------------------------------
%	frame 
%------------------------------------------------------------------------------

\begin{frame}
\frametitle{What Next}
\begin{itemize}
	\setlength{\itemsep}{5mm}
	\item Three period version is not intuitive. 
	\begin{itemize}
		\item I have made progress solving general numeric examples, but unsure how useful this is
	\end{itemize}
	\item More than two signal levels seems promising
	\begin{itemize}
		\item  I expect I will also need to resort to numeric examples 
	\end{itemize} 
	\item A continuous signal or just employers fully learning workers continuous ability measure 
	\begin{itemize}
		\item I started this and do not think there is a general solution. (Depends on $\theta$ distribution)
	\end{itemize}
	\item Promotions
	\item An Acemoglu and Pischke type utility shock 
\end{itemize}
\end{frame}

\begin{frame}
\frametitle{Questions and Concerns }
\begin{itemize}
	\setlength{\itemsep}{5mm}
	\item How to apply model to data?
	\item Are numerical examples and graphs useful?
	\item Trouble fitting this into literature 
	\item Are the assumptions reasonable? 
	
\end{itemize}
\end{frame}



\begin{frame}

\frametitle{The End }

\begin{center}
	
\begin{Huge}
\textbf{Thank You}
\end{Huge}
\end{center}
\end{frame}

%------------------------------------------------
% end doc
%------------------------------------------------
\end{document}

