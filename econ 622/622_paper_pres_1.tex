%-----------------------------------------------------------------------------------
%	PACKAGES AND OTHER DOCUMENT CONFIGURATIONS
%----------------------------------------------------------------------------------

\documentclass{beamer}


\newcommand{\Var}{\mathrm{Var}}

\newcommand{\Cov}{\mathrm{Cov}}

\newcommand{\plim}{\rightarrow_{p}}

\usepackage{apacite}

\usepackage{amsmath, amsfonts}
\usepackage{graphicx}
\usepackage{pdfpages}
\usepackage{bm}
\usepackage{listings}
\usepackage{multirow,array}
\usepackage{enumerate}
\usepackage{bbm}
\usepackage{subfig}
\usepackage{bbm}
\usepackage{multirow}



\usepackage{amssymb}

\usepackage{mathrsfs}
\usepackage{float}
\usepackage{booktabs}
\usepackage{color}
\usepackage{rotating}
\usepackage{amsthm}
\usepackage{multirow,array}
\usepackage{caption}
\usepackage{url}



\DeclareMathOperator*{\argmax}{arg\,max}
\DeclareMathOperator*{\argmin}{arg\,min}



% Expectation symbol
\newcommand{\E}{\mathrm{E}}
\newcommand{\V}{\mathrm{V}}
\newcommand{\N}{\mathcal{N}}
\newcommand{\R}{\mathbb{R}} 



%-----------------------------
% title stuff 
%-------------------------

\usepackage[utf8]{inputenc}



%Information to be included in the title page:
\title{Asymmetric Learning Model of Resume Building With Wage Rigidity and Costly Firings}
\author{Nathan Mather}
\institute{University of Michigan}
\date{2019}





%------------------------------------------------------------------------------
%	begin doc
%------------------------------------------------------------------------------

\begin{document}
	
	
	
	\frame{\titlepage}
	
	\section{Introduction}
	
	\begin{frame}
	\frametitle{Introduction}
	
	\framesubtitle{Current Asymmetric Leaning Models}
	\begin{itemize}
		\setlength{\itemsep}{3mm}
		\item Asymmetric Learning seems to match intuition for labor market
		\item Current models' set up seems odd 
		\item Schönberg (2007)
		\begin{itemize}
			\item Employer learns workers true ability 
			\item Outside firms receive noisy signal about each specific workers ability 
		\end{itemize}
	\end{itemize}
\end{frame}





\begin{frame}
\frametitle{Model Environment}
\framesubtitle{My Model}
\begin{itemize}
	\setlength{\itemsep}{1mm}
	\item Workers have ability $\theta \in [0,1]$ 
	\begin{itemize}
		\setlength{\itemsep}{1mm}
		\item Produce $\theta$ per period 		
	\end{itemize}
\item Workers supply labor inelastically
\item Employers receive a good signal $g$ w.p. $\theta$ and a bad signal $b$ w.p. $1-\theta$
\item Employers offer wages each period (no long term contracts).
\item Outside firms see length of employment (resume)
\begin{itemize}
	\setlength{\itemsep}{1mm}
	\item i.e. receive a signal $e$ or $f$ 
\end{itemize}
	\item There are many identical firms
	\begin{itemize}
		\item Zero profit condition
	\end{itemize} 
	\item Downward wage rigidity 
	\item Fixed cost to fire employee

\end{itemize}
\end{frame}




\begin{frame}
\frametitle{Variable Definitions}

\begin{center}
	\resizebox{.75\linewidth}{!}{\begin{tabular}{||c | c||} 
			\hline
			Variable & Meaning  \\ [0.5ex] 
			\hline\hline
			$\theta$ & Ability \\ 
			\hline 
			$g$ & Good Signal\\ 
			\hline
			$b$ & Bad Signal \\
			\hline
			$e$ & Employed Signal\\
			\hline
			$f$ & Fired Signal\\
			\hline
			$F_C$ &  Fixed Cost to Firing \\ 
			\hline
			
			$w_1$ &  Period 1 Wage \\ 
			\hline
			$w_g$ & Wage After Good Signal \\
			\hline
			$w_b$ & Wage After Bad Signal \\
			\hline
			$w_f$ & Wage for Fired Worker\\	\hline
			$\pi$ & Profits\\[1ex] 
			\hline
		\end{tabular}
	}
\end{center}

\end{frame}



\begin{frame}
\frametitle{Time Line}


	\begin{itemize}
			\setlength{\itemsep}{3mm}
		\item period 1 wage offers
		\item Workers produce output
		\item Workers send signal of ability 
		\item Employers decide who to fire
		\item Employers offer period 2 wages conditional on signals 
		\item Outside Firms offer wages conditional on resume 
		\item Workers take the best offer and work for one more period 
		\item Workers retire 
	\end{itemize}

\end{frame}



\begin{frame}
\frametitle{Outline}
\begin{itemize}
	\setlength{\itemsep}{10mm}
	\item Start simple and build up model
	\item First consider two period model with flexible wages 
	\item Second, add in wage rigidity, but with high fixed costs to firing
	\item Finally, solve the complete model with wage rigidity and moderate firing costs 

\end{itemize}

\end{frame}



\begin{frame}
\frametitle{Two Period Flexible Wage Model}
\framesubtitle{Optimal Second Period Wage Offers}

\begin{itemize}
	\setlength{\itemsep}{3mm}
	\item With flexible wages there is no reason to fire workers
	\item Outside firms will not offer more than expected output given 1 year of employment $\E[ \theta | e] = p(g)\E[\theta | g] + p(b)\E[ \theta | b] = E[\theta]$
	\begin{itemize}
		\item Employers can pay their good signal employees $w_g = \E[\theta]$
	\end{itemize}
	\item An outside offer below $w_g$ would only attract bad signal employees 
	
	\begin{itemize}
		\item Employer can offer $ w_b = \E[\theta |b]$ 
	\end{itemize}

	\item In equilibrium workers stay where they are

\end{itemize}
\end{frame}


\begin{frame}
\frametitle{Two Period Flexible Wage Model}
\framesubtitle{Optimal First Period Wage Offers}

\begin{itemize}
	\setlength{\itemsep}{3mm}
	\item Zero profit condition means that first period wages should give expected profit of zero
	\item In other words, pay period 1 worker their expected profits
	
	$$w_1 = \E[\theta] + p(g)(\E[\theta|g] - w_g) + p(b)(\E[\theta | b] - w_b) 
	$$

	or 
	$$ w_1 = \E[\theta] + p(g)(\E[\theta | g] - \E[\theta]) $$
\end{itemize}

\end{frame}


\begin{frame}
\frametitle{Two Period Flexible Wage Model}
\framesubtitle{Result}

\begin{itemize}
	\setlength{\itemsep}{10mm}
	\item $w_1 > w_g > w_b$
	\item Employers pay a premium for period 1 workers so that they can earn a profit with inside knowledge in period 2. 
	\item This does not seem to reflect reality
\end{itemize}

\end{frame}


\begin{frame}
\frametitle{Two Period Sticky Wage, High Firing Cost}

\begin{itemize}
	\setlength{\itemsep}{10mm}
	\item Next introduce wage rigidity 
	\item For now, assume the fixed cost of firing is so high it is never optimal to fire
\end{itemize}

\end{frame}



\begin{frame}
\frametitle{Two Period Sticky Wage, High Firing Cost}
\framesubtitle{Lemma 1}

\textbf{Lemma 1:} \text{With sticky wages $w_1 \in (\E[\theta|b], \E[\theta | g])$} \\~\\

\textbf{Proof:} If $w_1 \geq \E[\theta | g] $ employers will make negative profits. If  $w_1 \leq \E[\theta | b] $ employers make positive profits. \\~\\

\textbf{Corollary} $w_b = w_1$ because $w_b$ will be "stuck"
\end{frame}



\begin{frame}
\frametitle{Two Period Sticky Wage, High Firing Cost}
\framesubtitle{Equilibrium wages }

\begin{itemize}
	\setlength{\itemsep}{3mm}
	\item Given Lemma 1, $w_b = w_1$
	\item By the same logic as in the flexible model we get $w_g = \E[\theta|e] = \E[\theta]$

	\item The period one wage to give zero profits is $w_1 = \E[\theta] + p(g)(\E[\theta|g] - \E[\theta]) + p(b)(\E[\theta|b] - w_1)$
	\item Solving this gives $w_1 = \E[\theta]$ 
	\item $w_1 = w_g = w_b = \E[\theta]$
	\end{itemize}
\end{frame}

\begin{frame}
\frametitle{Two Period Sticky Wages}
\framesubtitle{Not firing is not an equilibrium}
\begin{itemize}
	\setlength{\itemsep}{3mm}
	\item Let the fixed cost $F_C < \E[\theta] - \E[\theta |b]$
	\item NOT firing workers is NOT an equilibrium. 
	\item If a firm unilaterally deviates and fires workers after a bad signal they receive positive profits 
	$$ \pi = p(g) (\E[\theta|g] - \E[\theta]) - p(b)F_C $$
	$$ > p(g)( \E[\theta|g] - E[\theta]) + p(b)(E[\theta|b] - E[\theta]) = 0 $$
	\item Given this, all firms have an incentive to deviate and fire bad signal employees 
\end{itemize}

\end{frame}


\begin{frame}
\frametitle{Two Period Sticky Wages}
\framesubtitle{When is firing bad workers an equilibrium }
\begin{itemize}
	\setlength{\itemsep}{3mm}
	\item If all firms are firing, I could try to unilaterally deviate and keep bad signal employees
	\item Need to pay them at least $w_1$ 
	\item Bad signal workers accept $w_1$ since leaving to another firm would expose them as bad signal worker
	\item Unilaterally keeping low wage workers would change profits by 
	$$ \Delta \pi = p(b)(F_C + \E[\theta|b] - w_1)$$
	\item If $F_C < w_1 - \E[\theta|b]$ firing bad signal employees is an equilibrium outcome
	
\end{itemize}

\end{frame}

\begin{frame}
\frametitle{Two Period Sticky Wages}
\framesubtitle{Equilibrium Wages}
\begin{itemize}
	\setlength{\itemsep}{3mm}
		\item Assume $F_C < w_1 - \E[\theta|b]$ so all employers fire low signal workers
		\item Since only good signal workers are employed $w_g = \E[\theta |e] = \E[\theta|g]$
		\item  Need to offer a period one wage to get zero profits 
		$$w_1 = \E[\theta] + p(g)(\E[\theta|g] - \E[\theta|g]) - p(b) F_C   $$
		or 
		$$= \E[\theta] - p(b) F_C$$
		\item Fired workers are identified as low signal so they get re-hired for $w_f = \E[\theta |b]$
		\item So we get $wf_f < w_1 < \E[\theta] < w_g$ 
	
\end{itemize}

\end{frame}


\begin{frame}
\frametitle{Two Period Sticky Wages}
\framesubtitle{Equilibria}

\begin{itemize}
	\setlength{\itemsep}{3mm}
	\item If $F_C > \E[\theta] - \E[\theta |b]$ 
	\begin{itemize}
		\item No one is fired \\~\\
	\end{itemize} 

	\item If  $F_C <\E[\theta] - p(b) F_C - \E[\theta |b] \implies F_C < \frac{\E[\theta] - \E[\theta|b]}{1+p(b)}$ 
	\begin{itemize}
		\item Bad signal employees are fired \\~\\
	\end{itemize}

	\item If $F_C \in \left( \frac{\E[\theta] - \E[\theta|b]}{1+p(b)}, \E[\theta] - \E[\theta |b] \right)$
	\begin{itemize}
		\item I believe there will be a mixed equilibrium where some bad signal workers are fired. 
		\item Not proven yet
	\end{itemize}
	
\end{itemize}

\end{frame}


\begin{frame}
\frametitle{Numerical Example}

\begin{itemize}
		\setlength{\itemsep}{2mm}
	\item $\theta \in \{0,\frac{1}{3}, \frac{2}{3}, 1\}$ with equal probability
	\item $F_C = 0.1$ 
	\item $\E[\theta] = \frac{1}{2}$
	\item $\E[ \theta | g] = \frac{7}{9}$
	\item $\E[ \theta | b] = \frac{2}{9}$
	\item Flexible wages 
	\begin{itemize}
			\setlength{\itemsep}{1mm}
		\item $w_1 = 0.6388$
		\item $w_b = \frac{2}{9}$
		\item $w_g = \frac{1}{2}$
	\end{itemize}
\item Sticky wages 
	\begin{itemize}
		\setlength{\itemsep}{1mm}
		\item $w_1 = 0.407$
		\item $w_f = \frac{2}{9}$
		\item $w_g = \frac{7}{9}$
	\end{itemize}
\end{itemize}

\end{frame}


\begin{frame}
\frametitle{What Next}
\begin{itemize}
	\setlength{\itemsep}{5mm}
	\item Three period version 
	\item Promotions
	\item Initial education signal correlated with ability 
	\item Exogenous separations 
	\item An Acemoglu and Pischke type utility shock 
\end{itemize}
\end{frame}

\begin{frame}
\frametitle{Questions and Concerns }
\begin{itemize}
	\setlength{\itemsep}{5mm}
	\item How to apply model to data?
	\item What aspects of labor market should it explain? 
	\item Are the assumptions reasonable? 
	\item Is game theory approach useful? 
	
\end{itemize}
\end{frame}


%------------------------------------------------
% end doc
%------------------------------------------------
\end{document}

