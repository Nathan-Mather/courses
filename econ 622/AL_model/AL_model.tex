%-----------------------------------------------------------------------------------
%	PACKAGES AND OTHER DOCUMENT CONFIGURATIONS
%----------------------------------------------------------------------------------

\documentclass[11pt]{article}

\usepackage[top=2cm, bottom=3cm, left=2cm, right=2cm]{geometry}
\setlength{\parskip}{1em}
\setlength{\parindent}{4em}
\linespread{1.25}

\newcommand{\Var}{\mathrm{Var}}

\newcommand{\Cov}{\mathrm{Cov}}

\newcommand{\plim}{\rightarrow_{p}}

\usepackage{apacite}

\usepackage{amsmath, amsfonts}
\usepackage{graphicx}
\usepackage{pdfpages}
\usepackage{bm}
\usepackage{listings}
\usepackage{multirow,array}
\usepackage{enumerate}
\usepackage{bbm}
\usepackage{subfig}
\usepackage{bbm}

\usepackage[latin1]{inputenc}

\usepackage{amssymb}

\usepackage{mathrsfs}
\usepackage{float}
\usepackage{booktabs}
\usepackage{color}
\usepackage{rotating}
\usepackage{amsthm}
\usepackage{multirow,array}
\usepackage{caption}
\usepackage{url}



\DeclareMathOperator*{\argmax}{arg\,max}
\DeclareMathOperator*{\argmin}{arg\,min}



% Expectation symbol
\newcommand{\E}{\mathrm{E}}
\newcommand{\V}{\mathrm{V}}
\newcommand{\N}{\mathcal{N}}
\newcommand{\R}{\mathbb{R}} 

%----------------------------------------------------------------------------------
%	TITLE AND AUTHOR(S)
%----------------------------------------------------------------------------------

\title{Asymmetric Learning Model of Resume Building With Wage Rigidity and Costly Firings} % The article title

\author{Nathan Mather} % The article author(s) 

\date{\today} % An optional date to appear under the author(s)

\renewcommand{\contentsname}{Table of Contents}
%----------------------------------------------------------------------------------
\begin{document}
	
	
	%------------------------------------------------------------------------------
	%	TABLE OF CONTENTS
	%------------------------------------------------------------------------------
	\maketitle % Print the title/author/date block
	
	\setcounter{tocdepth}{3} % Set the depth of the table of contents to show sections and subsections only
	
	\tableofcontents % Print the table of contents
	
	%------------------------------------------------------------------------------
	% Introduction 
	%------------------------------------------------------------------------------

	\section{Introduction}
	The purpose of this document is to clearly outline the two period version of my model. The main idea and focus of my model is to investigate how wages are related to tenure in an asymmetric learning environment where outside firms only observe a resume like signal. The two period version of my model has a lot in common with the two period Acemoglu and Picshke model of asymmetric learning and general training \cite{AP_1998}. I, however, am not focusing on training, and differ from there set up in some key ways. \par
	
	Employers in my model will only observe a noisy signal of worker ability. In the two period model this is essentially indistinguishable from learning the true type. A firm that receives a good signal will consider that worker to have a high \textit{expected} ability rather than simply knowing they have high ability. I believe this departure will have more significant implications in a 3 period or more model. The second difference from the Acemoglu and Picshke model is that I assume outside firms can observe an employees resume. That is,  they can tell after period one if a worker was fired or separated from their firm, or if the firm decided to keep them. This has implications for the two period model, and will allow for a more realistic framework for labor force signals in 3 or more periods. Finally, I consider the impact of sticky wages and costly firings.  \par 


	The results of the model so far suggest that in an otherwise frictionless market, asymmetric information of this type will tend to lower wages and increase wage dispersion with tenure. A market with asymmetric information and sticky wages, but with high fixed costs to firing, will see relatively constant and even wages over tenure. Finally, a market with wage rigidity but moderate costs to firing will see wage growth with tenure and wage cuts for fired workers. 



%----------------------------------------------------------------------------------
% Two period Model Basics 
%----------------------------------------------------------------------------------


	\section{ Two period Model}
	\subsection{outline}
	
	I will start by discussing a model without wage rigidity. I then consider what happens to this model if we introduce wage rigidity, but let the cost to firing by extremely high. Finally, I consider what happens as the cost to firing is lowered. The Final model is the most interesting. However, considering the simpler versions in turn is crucial for solving the final model and makes the implications of each additional assumption more apparent.  
	\subsection{Environment}
	
	This model lasts for two periods. In the first period no firm has any information abut individual workers. The following information about the distribution of workers is common knowledge to all firms. Workers have ability $\theta \in [0,1]$ with distribution $f(\theta)$. They produce $\theta$ per period and supply labor inelastically. They take the highest wage offer they receive, and in the event of a tie stay with their current employer or pick randomly if unemployed. \par
	
	In the second period firms receive a noisy signal about employee's ability. Workers with a ability $\theta$ send a good signal with probability $\theta$ and a bad signal with probability $1-\theta$. Employers offer wages for period 2 employees conditional on this signal. Firms can also fire employees for a fixed cost $F_C$. Outside firms observe employee's resume. That is, if a worker is employed or has been fired after period one. Outside firms offer wages conditional on these signals. employees decide where to work and produce for one more period before retiring. There are many competitive firms so this implies a zero profit condition on all firms. \par 
	
	The variables I will be using are summarized in the table below. 
	
	\begin{center}
		\resizebox{.75\linewidth}{!}{\begin{tabular}{||c | c||} 
				\hline
				Variable & Meaning  \\ [0.5ex] 
				\hline\hline
				$\theta$ & Ability \\ 
				\hline 
				$g$ & Good Signal\\ 
				\hline
				$b$ & Bad Signal \\
				\hline
				$e$ & Employed Signal\\
				\hline
				$f$ & Fired Signal\\
				\hline
				$F_C$ &  Fixed Cost to Firing \\ 
				\hline
				
				$w_1$ &  Period 1 Wage \\ 
				\hline
				$w_g$ & Wage After Good Signal \\
				\hline
				$w_b$ & Wage After Bad Signal \\
				\hline
				$w_e$ & Wage Offered to Worker With a Year of Employment\\
				\hline
				$w_f$ & Wage for Fired Worker\\	\hline
				$\pi$ & Profits\\[1ex] 
				\hline
			\end{tabular}
		}
	\end{center}
	
The time-line is also laid out below.

	\begin{itemize}
	\setlength{\itemsep}{1mm}
	\item period 1 wage offers
	\item Workers produce output
	\item Workers send signal of ability 
	\item Employers decide who to fire
	\item Employers offer period 2 wages conditional on signals 
	\item Outside Firms offer wages conditional on resume 
	\item Workers take the best offer and work for one more period 
	\item Workers retire 
\end{itemize}

\subsection{Flexible Wage Model}

First we will consider a model with totally flexible wages. In this model there is no reason to ever fire workers. Employers can simply lower wages to whatever level is profitable or induce a worker to quit. Note that in period 2 the expected output of employees who sent a good or bad signal is $\E[\theta|g]$ and $\E[\theta|b]$ respectively. To determine wages we begin in the second period. 

\textbf{Proposition 1:} Second period wages in the flexible model will be $w_g = \E[\theta]$ and $w_b = \E[\theta|b]$. No workers will switch firms.

These wages are an equilibrium because they are the lowest possible wages an employer can make, and the firm maximize profits by offering the lowest possible wages. Consider if employers offered their good signal employees $w_g = \E[\theta] - \epsilon$. An outside firm could offer a wage of $w_e > w_g$ to my workers with a year of experience. All of them would leave, yielding the outside firm an expected profit of $\E[\theta] - w_e $. This will be profitable for outside firms for any $w_e < \E[\theta]$ and so employers must offer at least $w_g = \E[\theta]$ to retain their good signal workers. \par 

Given that $w_g = \E[\theta]$ any outside offer $w_e < \E[\theta]$ will only attract bad signal employees. This implies outside firms will offer $w_e = \E[\theta|b]$. So, employers can offer their bad signal employees $w_b = \E[\theta|b]$. the intuition is that outside firms can either offer a high wage and get both good and bad signal workers, for which they would pay at most $\E[\theta]$, or they can offer a lower wage and attract only bad signal employees, for which they wold offer at most $\E[\theta|b]$. Employers only need to match these offers to retain their workers. The result of these wages is that workers cannot gain by leaving their employer and employers have no incentive to fire. So, there is no turn over. \par

\textbf{proposition 2:} First period wages in the flexible model will be $w_1 = \E[\theta] + p(g)(\E[\theta|g] - \E[\theta])$

Given the wages in period 2, in order to determine period one wages all we need to do is apply the zero profit condition. The period one wage will be equal to period one output plus expected profits in period two. 


$$0 = \E[\theta] - w_1 + p(g)(\E[\theta|g] - w_g) + p(b)(\E[\theta | b] - w_b) $$
	
$$ \implies w_1 = \E[\theta] + p(g)(\E[\theta|g] - w_g) + p(b)(\E[\theta | b] - w_b) 
 = \E[\theta] + p(g)(\E[\theta | g] - \E[\theta]) $$
 
 Putting this together we get $w_1 > w_g > w_b$. The intuition here is that employers are able to earn expected profits on good signal workers in the second period and break even on bad signal workers. So, they bid up the wage of first period workers until their profits are back to zero. The wage tenure pattern here, wages falling with time, is inconsistent with what we see in reality. We can see that asymmetric information in an otherwise frictionless market works to lower wages with tenure.
 
 \subsection{Wage rigidity with High Fixed Cost to Firing }
 
 Now we will introduce downward wage rigidity. By ``downward wage rigidity" I mean that wages cannot go down in period 2. We will also start by assuming that the fixed cost to firing employees, $F_C$, is so high that no firm fires anyone. This is partially to help solve the final model, but also is of some interest. Some firms may face extremely high fixed costs to firing through legal risk, difficulty building a case for firing workers, difficulty getting managers to fire employees, or strong unions. The implications for this on the impact of wages and tenure is of interest. The conclusion is stated below. 
 
 \textbf{Proposition 3:} With sticky wages and high $F_C$ we get  $w_1 = w_b = w_g =\E[\theta]$
 
 The logic here is that offering a wage lower than $\E[\theta]$ in the first period would lead to positive profits, but offering a wage larger than $\E[\theta]$ in the first period would lead to negative profits. To start the solution I make an observation about possible values for period one wages. 
 
 \textbf{Lemma 1:} Given the high firing costs, $w_1 \in \left[\E[\theta|b], \E[\theta] \right]$ 
 
 We know that the profit maximizing wage offers, without wage rigidity, for firms in period 
two will be the wages offered in the flexible mode. So, if the firm offers period one wages 
$w_1 \leq \E[\theta | b]$ they can offer the same wages as in the flexible model in period two 
and receive positive profits. So this will not be an equilibrium. If a firm offers first 
period wages $w_1 \geq \E[\theta]$ they will not be able to lower wages in the second period. Thus the firm earns profits $\pi = \E[\theta] - w_1 + p(g)(\E[\theta|g] - w1) + p(b)(\E[\theta|b] - w_1) = 2\cdot(\E[\theta] - w1) < 0 $. Using this Lemma, we can determine optimal period two wages. First, we know 

 \textbf{Corollary 1:} $w_b = w_1$ because $w_b$ will be ``stuck"

Firms will want to lower their wages for their bad workers in period two to $\E[\theta |b]$, but because of sticky wages they will not be able to lower it. Thus, their best course of action is to make it as low as possible. Which is $w_b = w_1$. \par 

Now for their good signal workers, the same logic as the flexible model applies. The profit maximizing wage offer will be $\E[\theta]$. Since by lemma 1  $w_1 \leq \E[\theta]$ we know this will be possible. Thus $w_g = \E[\theta]$. \par 

Now to find period 1 wages we simply apply the zero profit condition with sticky wages. This is equivalent to setting $w_1$ equal to expected revenue since wages are the only cost. i.e. 

$$ w_1 = \E[\theta] + p(g)(\E[\theta|g] - \E[\theta]) + p(b)(\E[\theta|b] - w_1) $$

using the fact that $p(g)\E[\theta|g] + p(b)\E[\theta|b] = \E[\theta]$ we get 

$$ w_1 = \E[\theta] + \E[\theta] - p(g)\E[\theta] - p(b)w_1$$

using the fact that $ \E[\theta] - p(g)\E[\theta] = p(b)\E[\theta]$
$$ \implies w_1 +p(b)w_1 = \E[\theta] + p(b)\E[\theta]
$$

$$ \implies w_1 = \E[\theta]$$

Putting this all together we get proposition 3 $w_1 = w_b = w_g =\E[\theta]$. The implication of this is that sticky wages make it impossible for firms to capitalize on their inside information in period 2. Thus they earn no profits in period two and period one wages are not bid up past expected revenue in period 1. We will show in the next section that if we give firms an avenue to rid themselves of bad workers, through firing, they will do so. \par 

But how high is ``High fixed cost"? At what $F_C$ is this no longer an equilibrium? In order for the above equilibrium to hold we need that no firm could earn a profit by deviating and firing their bad workers. If a firm unilaterally deviates and fires workers after a bad signal they receive profits 
$$ \pi = p(g) (\E[\theta|g] - \E[\theta]) - p(b)F_C $$

This will not be optimal when it is less than profits from not firing bad signal worker. i.e. when 
$$  p(g) (\E[\theta|g] - \E[\theta]) - p(b)F_C < p(g)( \E[\theta|g] - E[\theta]) + p(b)(E[\theta|b] - E[\theta]) = 0 $$  

Which occurs when 

$$F_C > \E[\theta] - \E[\theta |b]$$

Thus when  $F_C > \E[\theta] - \E[\theta |b]$ the equilibrium is to not to fire any workers and pay a constant wage of $\E[\theta]$.

\subsection{Equilibrium with Firing}

Next we consider when firing all bad workers is an equilibrium. First let's determine what the wages would be if all firms decided to fire their bad signal employees. 

\textbf{proposition 4:} Wages in an equilibrium where bad signal employees are fired will be $w_1 = \E[\theta] - p(b)F_c$ and $ w_g = \E[\theta|g]$ and $w_f = \E[\theta|b]$. 

If bad signal employees are all fired than the outside employers will know that workers with a resume of $e$ are all good signal workers. So $w_e = \E[\theta|e] = \E[\theta|g]$. Since any worker that hasn't been fired can receive this wage, firms will have to pay their good signal workers $w_g = \E[\theta | g]$. Bad signal workers will have been fired and identified as bad signal and so they will be rehired for a wage of $w_f = \E[\theta | b]$ by other firms. \par

Given these second period actions and wages the zero profit condition will give us optimal first period wages of 

		$$w_1 = \E[\theta] + p(g)(\E[\theta|g] - \E[\theta|g]) - p(b) F_C = \E[\theta] - p(b) F_C$$
		
These wages will be a stable equilibrium whenever firms cannot deviate and receive positive profits. If a firm deviates and keeps their bad signal employees they can pay them $w_1$. They will not leave for another firm because leaving would identify them as a low skilled worker and yield them $\E[\theta|b] < w_1$. Therefore keeping bad signal employees gives 

$$\pi_{keep} = \E[\theta] - w1 + p(g)(\E[\theta|g] - w_g) + p(b)(\E[\theta|b] - w_1 ) = \E[\theta] - w_1 + p(b)(\E[\theta|b] - w_1)$$

compared to firing them and receiving 
$$\pi_{fire} = \E[\theta] - w1 + p(g)(\E[\theta|g] - w_g) - p(b)F_C $$

and so firing workers is optimal for all firms when 

$$ F_C <  w_1 - \E[\theta|b] $$

Given the $w_1$ we found above this implies firing all bad signal employees is optimal when 

$$F_C <\E[\theta] - p(b) F_C - \E[\theta |b] \implies F_C < \frac{\E[\theta] - \E[\theta|b]}{1+p(b)}$$

This equilibrium implies $w_f < w_1 < \E[\theta] < w_g$. The intuition here is that in period two firms either make 0 expected profits on good signal workers or have to pay to fire bad signal workers. Given this, workers in the first period are paid below their expected first period output. 
 
\subsection{Mixed Equilibrium}

If the fixed cost is $F_C \in \left( \frac{\E[\theta] - \E[\theta|b]}{1+p(b)}, \E[\theta] - \E[\theta |b] \right)$ than we will have a mixed equilibrium where firms fire a fraction $\delta_F$ of their bad signal employees.\par 

Note that in a mixed equilibrium firms would need to be indifferent between firing and keeping employees after a bad signal. This implies that the fixed cost to firing must be equivalent to the loss from holding on to a low signal worker. That is 

$$ F_C = w_1 - E[\theta|b]$$

Given that $\delta_F$ of the bad signal employees are fired, the wages required to keep a good signal employee are 

$$ w_g(\delta_F) = \frac{p(g) E[\theta|g] + p(g) (1-\delta_F) E[\theta|b] }{ p(g) +p(b)(1-\delta_F) } $$

This is just the expected output of a worker who is not fired. That is the only information a worker can prove to an outside employer and so this is the highest offer they could receive. Given that they are being offered this by their current employers, no good signal workers would leave for any outside offer and so low wage workers receive their "stuck" wage of $w_1$. \par 

Now given those period 2 wages the period 1 wages will satisfy the zero profit condition. 

$$ w_1 = E[\theta] + p(g) \left( E[\theta | g]  - w_g(\delta_F) \right) - p(b) F_C$$

Using these equations we can get a closed form solution for the fraction of bad signal workers fired $\delta_F$. 

$$ \delta_F = 1- \frac{p(g) \left[ E[\theta] - (1 + p(b))F_C - E[\theta|b] \right]}{p(b) \left[ -E[\theta] + (1+p(b))F_C + (1+p(g))E[\theta|b] - p(g)E[\theta|g] \right]}$$





%----------------------------------------------------------------------------------
% 2 period model with eogenous seperation. 
%----------------------------------------------------------------------------------

% I am going to try this outlined in the order that I think is better for presentation 
% can maybe go back and change the first section to match later 

\section{Two Period Model With Exogenous separation }

Now lets assume that a fraction $\delta$ of workers just exogenously separate after period one. We assume that employers cannot distinguish between someone who was fired or exogenously separate. This means there is only one wage offer for all unemployed workers in period 2. Employers can still offer different wages to current employees of other firms. 

\subsection{Flexible wages}

In the unrestricted model of asymmetric information exogenous separation significantly changes the interpretation of the result. Let $Q$ denote the fraction of workers that don't separate and quit. The wage offer to an unemployed worker is   

$$ w_u = (\delta \E[\theta] + Q \E[\theta | \text{worker quit}])/(\delta + Q)$$

We also know that employers will not pay more than the expected output of a worker, so 

$$ w_b = \E[\theta | b] $$

This implies that all bad signal employees quit since if $\delta > 0$ 

$$ \E[\theta|b] < \frac{ \delta \E[\theta] + (1-\delta)p(b)( \E[\theta | b])}{\delta + (1-\delta)p(b)}$$

where the Right hand side is the lowest possible $w_u$. If all bad signal employees quite than anyone still employed is identified as high ability and so we get a high signal wage 

$$ w_g = \E[\theta|g]$$

and an unemployed wage offer of  

$$ w_u =  \frac{ \delta \E[\theta] + (1-\delta)p(b)( \E[\theta | b])}{\delta + (1-\delta)p(b)}$$

and all bad signal employees quit. 

This gives a first period wage offer of 

$$w_1 = \E[\theta]$$



\subsection{sticky wages}

The outcome above seems unrealistic because large real wage cuts don't typically happen. Moreover, if an employer is dissatisfied with an employee they are likely to fire them, not cut their wage an induce them to quit.In fact, in the above model no one is ever fired, and with the logic there firings don't really make sense. To correct this I first introduce sticky wages. Now employers can't cut wages but they can fire their employees.\par 

by the same logic as without exogenous separation we know $w_1 > \E[\theta | b]$ so employers will fire bad signal workers. This actually gives us the same outcome as the flexible wage model but now all bad signal employees are fired rather than quitting. 

\subsection{fixed cost to firing}

\subsubsection{All bad signal workers fired}

The sticky wage result of all bad signal workers being fired at lest acknowledges the existence of firing, but it may not be completely satisfactory either. Certainly employers don't fire employees as soon as their expected output drops just below their wage? Firing employees and creating turnover can be difficult. \par 

next we incorporate a fixed cost to firing employees. If the fixed cost is very low, it will not alter the employer's behavior in period two. They simply pay the fixed cost and fire their bad signal employees. However, this does alter the period one wage offer. Now that hiring an employee means I may have to pay a cost to fire them, their marginal benefit is decreased and so their wage offer is lower. Specifically we get

$$ w_1 = \E[\theta] + (1- /delta)p(b)F_C$$

firms continue to do this as long as the fixed cost to firing is less than the loss from holding onto a bad signal worker. Since they can't lower their wage, keeping a bad signal employee would cost $\E[\theta|b] - w_1$. Thus, paying the fixed cost and firing all bad signal workers is stable as long as 

$$ w_1 - \E[\theta|b] > F_C$$

$$ \implies  \E[\theta] + (1- \delta)p(b)F_C -  \E[\theta|b] > F_C $$

$$ \implies f_C < \frac{\E[\theta] -  \E[\theta|b]}{1 + p(b)(1-\delta)} $$

\subsubsection{No workers fired} 
If the fixed costs become sufficiently large, it is clear that firms will not fire any employees. In this case firms must pay their good signal employees enough to keep them from getting poached. 

$$ w_g = E[\theta | e] =\E[\theta]$$

They will pay their bad signal employees as little as possible 
$$ w_b = w_1$$ 

unemployed workers get their expected output 
$$ w_u = E[\theta] $$

this gives a first period wage through the zero profit condition 
$$w_1 = \E[\theta] + (1-\delta)(p(g)(\E[\theta|g] - \E[\theta]) + p(b)(\E[\theta|b] - w_1))$$


using the fact that $p(g)\E[\theta|g] + p(b)\E[\theta|b] = \E[\theta]$ we get 

$$ w_1 = \E[\theta] +  (1-\delta)(\E[\theta] - p(g)\E[\theta] - p(b)w_1)$$

using the fact that $ \E[\theta] - p(g)\E[\theta] = p(b)\E[\theta]$

$$ \implies  w_1 = \E[\theta] +  (1-\delta)p(b)(\E[\theta] - w_1) $$
$$ \implies w_1 +(1-\delta)p(b)w_1 = \E[\theta] + (1-\delta)p(b)\E[\theta]$$

$$ \implies w_1 = \E[\theta]$$

So all workers get paid $\E[theta]$ no one is fired and also no workers are induced to quit since quiting gets them the same wage. Firms have inside information but they are too restricted to do anything with it. They can't lower their bad employees wages or fire them. \par 

This equilibrium will be stable whenever it is more costly to fire an employee then to keep them at this wage. That is when 

$$ F_C >  w_1 - \E[\theta|b] = \E[\theta] - \E[\theta|b]$$


\subsubsection{mixed equilibrium}

Now consider the case between the two extremes. when 

$$ F_C \in \left[  \frac{\E[\theta] - \E[\theta]}{1 + p(b)(1-\delta)} , \E[\theta] - \E[\theta|b] \right] $$

In this case employers will fire a fraction of their bad signal workers $\delta_F$ until they are indifferent between firing and keeping them 
$$  F_C = w_1 - E[\theta|b]$$ 

Good signal employees need to be paid the expected output of an average worker employee in period 2 

$$ w_g(\delta_F) = \frac{p(g)\E[\theta|g] + p(b)(1-\delta_F) \E[\theta|b] }{p(g) + p(b)(1-\delta_F)} $$

bad signal employees are paid the lowest wage possible $ w_b = w_1$ and unemployed get their expected output 

$$ w_u = \delta\E[\theta] + (1-\delta)\delta_F\E[\theta | b]$$

by the zero profit condition we get the first period wages 

$$w_1 = \E[\theta] + (1-\delta)  \bigg( p(g) \Big(\E[\theta|g] - w_g(\delta_F)  \Big) - p(b)F_C  \bigg) $$

















%----------------------------------------------------------------------------------
% Three Period Model
%----------------------------------------------------------------------------------


\section{Three Period Model}

Now we will extend the model to three periods. Following the notation above $w_xy$ refers to wages after sending signals of $x$ and $y$ to an employer in period 1 and 2 respectively. 

\subsection{Flexible Wages}

\subsubsection{Period 3 wages }

With completely flexible wages the logic follows that of the two period model. We start from the final period. Given that Everyone is employed, outside firms has essentially no information about individual employees and so employing firms need only pay the expected output of the average worker to their best employees. 

$$ w_{gg} = \E[\theta|ee] = \E[\theta]$$

For workers who have sent a good and a bad signal, employers must offer a wage equal to the average output of workers at least as good. That is, 

$$w_{gb} = w_{bg} = \E[\theta| ee, s_1 s_2 \notin gg] = \frac{2 \cdot p(gb)\E[\theta|gb] + p(bb) \E[\theta|bb]}{2 \cdot p(gb) + p(bb)} $$

The worst signal employees will receive just their expected output 

$$ w_{bb} = \E[\theta|bb]$$

\subsubsection{Period 2 wages }

In period 1, to retain good signal workers, employers need to offer the expected profit of an average period 1 worker. Outside Firms will obtain a worker with ability $\E[\theta]$ and receive 1 signal before retirement. Thus, the value of this average worker is identical to the value of new hire in the two period model. Thus 

$$ w_g = \E[\pi|e] = \E[\theta|e] + p(g|e)(\E[\theta|eg] - E[\theta|e])$$

Given that no one is fired, being employed for a period tells the outside firm nothing. So this is equivalent to 

$$ w_g = \E[\theta] + p(g)(\E[\theta|g] - E[\theta]) $$

Low signal workers, for the same logic as throughout the model, are paid their expected profits. 

$$ w_b = \E[\pi|b] = \E[\theta|b] + p(g|b)(\E[\theta|gb] - w_{gb}) + p(b|g)(\E[\theta|bb] - w_{bb})$$


\subsubsection{Period 1 wages }

Now period 1 wages are the wages that satisfy the zero profit condition 

$$w_1 = \E[\theta] + p(g) (\E[\theta|g] - w_g) + p(gg)(\E[\theta|gg] - w_{gg}) + 2\cdot p(gb)(\E[\theta|gb] - w_{gb})$$


\subsection{Finding Equilibrium cutoffs}



%------------------------------------------------------------------------------
% bib
%------------------------------------------------------------------------------
\bibliographystyle{apacite}
\bibliography{References}
	
	%------------------------------------------------
	% end doc
	%------------------------------------------------
\end{document}





