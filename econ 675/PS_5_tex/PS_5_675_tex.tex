%-----------------------------------------------------------------------------------
%	PACKAGES AND OTHER DOCUMENT CONFIGURATIONS
%----------------------------------------------------------------------------------

\documentclass[11pt]{article}

\usepackage[top=2cm, bottom=3cm, left=2cm, right=2cm]{geometry}

\setlength{\parindent}{0in}

\newcommand{\Var}{\mathrm{Var}}

\newcommand{\Cov}{\mathrm{Cov}}

\newcommand{\plim}{\rightarrow_{p}}

\usepackage{amsmath, amsfonts}
\usepackage{graphicx}
\usepackage{pdfpages}
\usepackage{bm}
\usepackage{listings}
\usepackage{multirow,array}
\usepackage{enumerate}
\usepackage{bbm}


\usepackage[latin1]{inputenc}

\usepackage{amssymb}

\usepackage{mathrsfs}
\usepackage{float}
\usepackage{booktabs}
\usepackage{color}
\usepackage{rotating}
\usepackage{amsthm}
\usepackage{multirow,array}
\usepackage{caption}
\usepackage{url}


\DeclareMathOperator*{\argmax}{arg\,max}
\DeclareMathOperator*{\argmin}{arg\,min}



% Expectation symbol
\newcommand{\E}{\mathrm{E}}
\newcommand{\V}{\mathrm{V}}
\newcommand{\N}{\mathcal{N}}
\newcommand{\R}{\mathbb{R}} 

%----------------------------------------------------------------------------------
%	TITLE AND AUTHOR(S)
%----------------------------------------------------------------------------------
\title{Econ 675 Assignment 3} % The article title


\author{Nathan Mather\thanks{Shouts out to Ani  for the help with this. Could not have done it without you! } } % The article author(s) 

\date{\today} % An optional date to appear under the author(s)


%----------------------------------------------------------------------------------
\begin{document}
	
%------------------------------------------------------------------------------
%	TABLE OF CONTENTS & LISTS OF FIGURES AND TABLES
%------------------------------------------------------------------------------
\maketitle % Print the title/author/date block

\setcounter{tocdepth}{2} % Set the depth of the table of contents to show sections and subsections only

\tableofcontents % Print the table of contents


%-------------------------------------------------------------
% Question 1 
%-------------------------------------------------------------

\section{Question 1: Many Instruments Asymptotics}




%-------------------------------------------------------------
% Question 2
%-------------------------------------------------------------

\section{Question 2: Weak Instruments Simulations}

\begin{center}
	
	\centering
	
	\textbf{Results for $n \gamma^2$ = 0}\par\medskip
	\scalebox{1}{
		% latex table generated in R 3.5.1 by xtable 1.8-3 package
% Tue Nov 13 14:50:16 2018
\begin{tabular}{llrrrrr}
  \hline
reg\_type & variable & mean & st.dev & quant .1 & quant .5 & quant .9 \\ 
  \hline
ols & estimate & 1.00 & 0.01 & 0.99 & 1.00 & 1.01 \\ 
  ols & std.error & 0.01 & 0.00 & 0.01 & 0.01 & 0.01 \\ 
  ols & rej & 1.00 & 0.00 & 1.00 & 1.00 & 1.00 \\ 
  2sls & estimate & 1.37 & 1.87 & 0.79 & 1.00 & 1.78 \\ 
  2sls & std.error & 26.80 & 108.52 & 0.07 & 0.20 & 6.78 \\ 
  2sls & rej & 0.64 & 0.48 & 0.00 & 1.00 & 1.00 \\ 
  2sls & f\_stat & 0.94 & 1.16 & 0.01 & 0.66 & 2.37 \\ 
   \hline
\end{tabular}

	}
\end{center}

\begin{center}
	
	\centering
	
	\textbf{Results for $n \gamma^2$ = 0.25}\par\medskip
	\scalebox{1}{
		% latex table generated in R 3.5.1 by xtable 1.8-3 package
% Tue Nov 13 14:50:16 2018
\begin{tabular}{llrrrrr}
  \hline
reg\_type & variable & mean & st.dev & quant .1 & quant .5 & quant .9 \\ 
  \hline
ols & estimate & 1.00 & 0.01 & 0.99 & 1.00 & 1.01 \\ 
  ols & std.error & 0.01 & 0.00 & 0.01 & 0.01 & 0.01 \\ 
  ols & rej & 1.00 & 0.00 & 1.00 & 1.00 & 1.00 \\ 
  2sls & estimate & 1.20 & 4.84 & -1.20 & 0.68 & 4.11 \\ 
  2sls & std.error & 38.99 & 126.72 & 0.18 & 1.22 & 94.19 \\ 
  2sls & rej & 0.26 & 0.44 & 0.00 & 0.00 & 1.00 \\ 
  2sls & f\_stat & 0.96 & 1.50 & 0.00 & 0.46 & 2.59 \\ 
   \hline
\end{tabular}

	}
\end{center}

\begin{center}
	
	\centering
	
	\textbf{Results for $n \gamma^2$ = 9}\par\medskip
	\scalebox{1}{
		% latex table generated in R 3.5.1 by xtable 1.8-3 package
% Tue Nov 13 14:50:16 2018
\begin{tabular}{llrrrrr}
  \hline
reg\_type & variable & mean & st.dev & quant .1 & quant .5 & quant .9 \\ 
  \hline
ols & estimate & 0.96 & 0.02 & 0.94 & 0.96 & 0.98 \\ 
  ols & std.error & 0.02 & 0.00 & 0.01 & 0.02 & 0.02 \\ 
  ols & rej & 1.00 & 0.00 & 1.00 & 1.00 & 1.00 \\ 
  2sls & estimate & -0.15 & 0.51 & -0.72 & -0.03 & 0.25 \\ 
  2sls & std.error & 0.53 & 0.59 & 0.19 & 0.34 & 1.08 \\ 
  2sls & rej & 0.08 & 0.27 & 0.00 & 0.00 & 0.00 \\ 
  2sls & f\_stat & 9.68 & 5.82 & 2.57 & 9.32 & 18.58 \\ 
   \hline
\end{tabular}

	}
\end{center}

\begin{center}
	
	\centering
	
	\textbf{Results for $n \gamma^2$ = 99}\par\medskip
	\scalebox{1}{
		% latex table generated in R 3.5.1 by xtable 1.8-3 package
% Tue Nov 13 23:33:37 2018
\begin{tabular}{llrrrrr}
  \hline
reg\_type & variable & mean & st.dev & quant .1 & quant .5 & quant .9 \\ 
  \hline
ols & estimate & 0.67 & 0.03 & 0.62 & 0.67 & 0.71 \\ 
  ols & std.error & 0.03 & 0.00 & 0.03 & 0.03 & 0.04 \\ 
  ols & rej & 1.00 & 0.00 & 1.00 & 1.00 & 1.00 \\ 
  2sls & estimate & -0.01 & 0.11 & -0.15 & -0.00 & 0.11 \\ 
  2sls & std.error & 0.10 & 0.02 & 0.08 & 0.10 & 0.14 \\ 
  2sls & rej & 0.05 & 0.21 & 0.00 & 0.00 & 0.00 \\ 
  2sls & f\_stat & 100.93 & 24.69 & 71.05 & 99.09 & 133.35 \\ 
   \hline
\end{tabular}

	}
\end{center}




%-------------------------------------------------------------
% Question 3
%-------------------------------------------------------------
\section{Question 3: Weak Instrument - Empirical Study}

\subsection{Question 3.1}

\begin{center}
	
	\centering
	
	\textbf{Results from R}\par\medskip
	\scalebox{1}{
		% latex table generated in R 3.5.1 by xtable 1.8-3 package
\begin{tabular}{llrr}
  \hline
model & term & estimate & std.error \\ 
  \hline
OLS 1 & educ & 0.06 & 0.00 \\ 
  OLS 2 & educ & 0.06 & 0.00 \\ 
  2sls 1 & educ & 0.09 & 0.02 \\ 
  2sls 2 & educ & 0.06 & 0.03 \\ 
   \hline
\end{tabular}

	}
\end{center}



%------------------------------------------------
% end doc
%------------------------------------------------
\end{document}
