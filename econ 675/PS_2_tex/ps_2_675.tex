%-----------------------------------------------------------------------------------
%	PACKAGES AND OTHER DOCUMENT CONFIGURATIONS
%----------------------------------------------------------------------------------



\documentclass[11pt]{article}

\usepackage[top=2cm, bottom=3cm, left=2cm, right=2cm]{geometry}

\setlength{\parindent}{0in}

\newcommand{\Var}{\mathrm{Var}}

\newcommand{\Cov}{\mathrm{Cov}}

\newcommand{\plim}{\rightarrow_{p}}

\usepackage{amsmath, amsfonts}
\usepackage{graphicx}
\usepackage{pdfpages}
\usepackage{bm}
\usepackage{listings}

% Expectation symbol
\newcommand{\E}{\mathrm{E}}
\newcommand{\V}{\mathrm{V}}

%----------------------------------------------------------------------------------
%	TITLE AND AUTHOR(S)
%----------------------------------------------------------------------------------

\title{Econ 675 Assignment 1} % The article title


\author{Nathan Mather} % The article author(s) 

\date{\today} % An optional date to appear under the author(s)


%----------------------------------------------------------------------------------
\begin{document}
	
%------------------------------------------------------------------------------
%	TABLE OF CONTENTS & LISTS OF FIGURES AND TABLES
%------------------------------------------------------------------------------
\maketitle % Print the title/author/date block

\setcounter{tocdepth}{2} % Set the depth of the table of contents to show sections and subsections only

\tableofcontents % Print the table of contents


%-------------------------------------------------------------
% Question 1 
%-------------------------------------------------------------
\section{Kernal Density Estimation}
\subsection{Part 1}

Start by noting that 

$$ \hat{f}^{(r)}(x) = \frac{(-1)^s}{nh^{1+s}} \sum_{i=1}^{n}k^{(s)} \left( \frac{{x}_i - x}{h} \right) 
$$

Now taking the expectation of $\hat{f}^{(r)}(x)$ that we can apply the linearity of expectations to move the expectation inside the sum. Then we can use the i.i.d. assumption to show the sum is just n times the expectation. This leaves us with 

$$  \E[\hat{f}^{(r)}(x)] = \E \left[ \frac{(-1)^s}{h^{1+s}} k^{(s)} \left( \frac{{x}_i - x}{h} \right)  \right]
= \int_{-\infty}^{\infty} \frac{(-1)^s}{h^{1+s}} k^{(s)} \left( \frac{z - x}{h} \right)f(z)dz
$$



Where the second equality is just by the definition of the expectation. Next we use integration by parts. Note that 

$$\int_{-\infty}^{\infty} \frac{(-1)^s}{h^{1+s}} k^{(s)} \left( \frac{z - x}{h} \right)f(z)dz = -\int_{-\infty}^{\infty} \frac{(-1)^s}{h^{s}} k^{(s-1)} \left( \frac{z - x}{h} \right)f^{(1)}(z)dz
$$

Iterating this s times gives us

$$\int_{-\infty}^{\infty} \frac{(-1)^s}{h^{1+s}} k^{(s)} \left( \frac{z - x}{h} \right)f(z)dz
=  (-1)^s \int_{-\infty}^{\infty} \frac{(-1)^s}{h} k \left( \frac{z - x}{h} \right)f^{(s)}(z)dz
= \int_{-\infty}^{\infty} \frac{1}{h} k \left( \frac{z - x}{h} \right)f^{(s)}(z)dz
$$

Next we apply change of variables. let $u = \frac{z - x}{h}$ Note that $du=\frac{1}{h}dz$ so we get 

$$ \int_{-\infty}^{\infty}  k(u)f^{(s)}(x+hu)du $$

Next we Taylor expand $f^{(s)}(x+hu)$ to the $P^{th}$ order about $x$. Recall from properties of the kernal estimator that $ \int_{-\infty}^{\infty}k(u)du = 1$ and that $ \int_{-\infty}^{\infty}k(u)u^jdu = 0$ for all $j\neq p$ This gives us

$$ f^{(r)}(x) +\frac{1}{P!}f^{(s+P)}(x)h^P\int_{-\infty}^{\infty}k(u)u^pdu +o(h^P)
= f^{(r)}(x) +\frac{1}{P!}f^{(s+P)}(x)h^p \mu_P(k) +o(h^P)
$$

which is the desired result. 
\\ \\ 

Now for the variance recall again that 

$$ \hat{f}^{(r)}(x) = \frac{(-1)^s}{nh^{1+s}} \sum_{i=1}^{n}k^{(s)} \left( \frac{{x}_i - x}{h} \right) 
$$

So by the i.i.d. assumption we can get that 

$$ \V \left(\hat{f}^{(r)}(x) \right) = \frac{1}{nh^{2+2s}} \V \left( k^{(s)} \left( \frac{{x}_i - x}{h} \right)  \right)
$$

%------------------------------------------------
% end doc
%------------------------------------------------
\end{document}


