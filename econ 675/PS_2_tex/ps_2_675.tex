%-----------------------------------------------------------------------------------
%	PACKAGES AND OTHER DOCUMENT CONFIGURATIONS
%----------------------------------------------------------------------------------



\documentclass[11pt]{article}

\usepackage[top=2cm, bottom=3cm, left=2cm, right=2cm]{geometry}

\setlength{\parindent}{0in}

\newcommand{\Var}{\mathrm{Var}}

\newcommand{\Cov}{\mathrm{Cov}}

\newcommand{\plim}{\rightarrow_{p}}

\usepackage{amsmath, amsfonts}
\usepackage{graphicx}
\usepackage{pdfpages}
\usepackage{bm}
\usepackage{listings}

% Expectation symbol
\newcommand{\E}{\mathrm{E}}
\newcommand{\V}{\mathrm{V}}
\newcommand{\N}{\mathcal{N}}

%----------------------------------------------------------------------------------
%	TITLE AND AUTHOR(S)
%----------------------------------------------------------------------------------

\title{Econ 675 Assignment 1} % The article title


\author{Nathan Mather} % The article author(s) 

\date{\today} % An optional date to appear under the author(s)


%----------------------------------------------------------------------------------
\begin{document}
	
%------------------------------------------------------------------------------
%	TABLE OF CONTENTS & LISTS OF FIGURES AND TABLES
%------------------------------------------------------------------------------
\maketitle % Print the title/author/date block

\setcounter{tocdepth}{2} % Set the depth of the table of contents to show sections and subsections only

\tableofcontents % Print the table of contents


%-------------------------------------------------------------
% Question 1 
%-------------------------------------------------------------
\section{Kernal Density Estimation}
\subsection{Part 1}

Start by noting that 

$$ \hat{f}^{(s)}(x) = \frac{(-1)^s}{nh^{1+s}} \sum_{i=1}^{n}k^{(s)} \left( \frac{{x}_i - x}{h} \right) 
$$

Now taking the expectation of $\hat{f}^{(s)}(x)$ that we can apply the linearity of expectations to move the expectation inside the sum. Then we can use the i.i.d. assumption to show the sum is just n times the expectation. This leaves us with 

$$  \E[\hat{f}^{(s)}(x)] = \E \left[ \frac{(-1)^s}{h^{1+s}} k^{(s)} \left( \frac{{x}_i - x}{h} \right)  \right]
= \int_{-\infty}^{\infty} \frac{(-1)^s}{h^{1+s}} k^{(s)} \left( \frac{z - x}{h} \right)f(z)dz
$$



Where the second equality is just by the definition of the expectation. Next we use integration by parts. Note that 

$$\int_{-\infty}^{\infty} \frac{(-1)^s}{h^{1+s}} k^{(s)} \left( \frac{z - x}{h} \right)f(z)dz = -\int_{-\infty}^{\infty} \frac{(-1)^s}{h^{s}} k^{(s-1)} \left( \frac{z - x}{h} \right)f^{(1)}(z)dz
$$

Iterating this s times gives us

$$\int_{-\infty}^{\infty} \frac{(-1)^s}{h^{1+s}} k^{(s)} \left( \frac{z - x}{h} \right)f(z)dz
=  (-1)^s \int_{-\infty}^{\infty} \frac{(-1)^s}{h} k \left( \frac{z - x}{h} \right)f^{(s)}(z)dz
= \int_{-\infty}^{\infty} \frac{1}{h} k \left( \frac{z - x}{h} \right)f^{(s)}(z)dz
$$

Next we apply change of variables. let $u = \frac{z - x}{h}$ Note that $du=\frac{1}{h}dz$ so we get 

$$ \int_{-\infty}^{\infty}  k(u)f^{(s)}(x+hu)du $$

Next we Taylor expand $f^{(s)}(x+hu)$ to the $P^{th}$ order about $x$. Recall from properties of the kernal estimator that $ \int_{-\infty}^{\infty}k(u)du = 1$ and that $ \int_{-\infty}^{\infty}k(u)u^jdu = 0$ for all $j\neq p$ This gives us

$$ f^{(s)}(x) +\frac{1}{P!}f^{(s+P)}(x)h^P\int_{-\infty}^{\infty}k(u)u^pdu +o(h^P)
= f^{(s)}(x) +\frac{1}{P!}f^{(s+P)}(x)h^p \mu_P(k) +o(h^P)
$$

which is the desired result. 
\\ \\ 

Now for the variance recall again that 

$$ \hat{f}^{(s)}(x) = \frac{(-1)^s}{nh^{1+s}} \sum_{i=1}^{n}k^{(s)} \left( \frac{{x}_i - x}{h} \right) 
$$

So by the i.i.d. assumption we can get that 

$$ \V \left(\hat{f}^{(s)}(x) \right) = \frac{1}{nh^{2+2s}} \V \left( k^{(s)} \left( \frac{{x}_i - x}{h} \right)  \right)
$$

\begin{align}
\V \left(\hat{f}^{(s)}(x) \right) &= \frac{1}{nh^{2+2s}} \V \left( k^{(s)} \left( \frac{{x}_i - x}{h} \right)  \right)\\
 &= \frac{1}{n2h^{2+2s}} \E \left[\left( k^{(s)} \left( \frac{{x}_i - x}{h} \right)  \right)^2 \right] - \frac{1}{nh^{2+2s}} \E \left[\left( k^{(s)} \left( \frac{{x}_i - x}{h} \right)  \right)^2 \right]^2 \\
  &= \frac{1}{nh^{2+2s}} \E \left[\left( k^{(s)} \left( \frac{{x}_i - x}{h} \right)  \right)^2 \right] - \frac{1}{n}\left( \frac{1}{h^{1+s}} \E \left[\left( k^{(s)} \left( \frac{{x}_i - x}{h} \right)  \right)^2 \right] \right)^2 \\
  &=  \frac{1}{nh^{2+2s}} \int_{-\infty}^{\infty}k^{(s)} \left( \frac{{x}_i - x}{h} \right)^2 f(z)dz +   \frac{1}{nh^{2+2s}} f^{(n)}(X)^2 \\
  &= \frac{1}{nh^{1+2s}} \int_{-\infty}^{\infty} k^{(s)}(u)^2f(x+hu)du + o\left(\frac{1}{nh^{2+2s}} \right)\\
  &= \frac{1}{nh^{1+2s}} \cdot \vartheta_s(K) + o\left(\frac{1}{nh^{2+2s}} \right)
\end{align}

\subsection{part 2}

We start with the following AMISE
$$ AIMSE[h] = \int \left[ \left( h_n^P \cdot \mu_P(K) \cdot \frac{f^{(P+s)}(x)}{P!}   \right)^2 + \frac{1}{nh_n^{1+2s}} \cdot \vartheta_s(K) \cdot f(x) \right]dx
$$

Using the $\vartheta$ notation so $\vartheta_{P+s}(f) = \int(f^{(P+s)}(x))^2$ and recalling that f(x) integrates to 1 we can rewrite this as 
$$ =  h_n^{2P} \left( \frac{\mu_P(K)}{P!} \right)^2 \vartheta_{P+s}(f) + \frac{\vartheta_s(K)}{nh_n^{1+2s}}
$$
Now taking first order conditions and solving for h 
$$ \frac{d}{dh}AIMSE[h] =  2Ph_n^{2p-1} \left( \frac{\mu_P(K)}{P!} \right)^2 \vartheta_{P+s}(f) -(1+2s) \frac{\vartheta_s(K)}{nh_n^{2+2s}} =0
$$
$$ \implies 2Ph^{1+2P+2s}\left( \frac{\mu_P(K)}{P!} \right)^2 \vartheta_{P+s}(f) = (1+2s) \frac{\vartheta_s(K)}{n}
$$

Thus, we get the AIMSE-optimal bandwidth choice.
$$h_{AIMSE_s} = \left[ \frac{(2s+1)(P!)^2}{2P} \frac{\vartheta_s(K)}{\vartheta_{s+P}(f) \cdot \mu_P(K)^2} \frac{1}{n} \right]^{\frac{1}{1+2P+2s}}$$

Least squares cross-validation is a fully automatic data-driven method of selecting the smoothing parameter h. THis is based on the principle of selecting bandwidth that minimizes the integrated squared error of the resulting estimate.The estimate used is 

$$ \hat{h}_{CV} = \arg \min_{h} \frac{1}{n^2h} \sum_{i=1}^n \sum_{j=1}^n \bar{k} \left(\frac{X_i - X_j}{h}\right) - \frac{2}{n(n-1)h}\sum_{i=1}^n\sum_{j=1, i\neq j}^nk\left(\frac{X_i - X_j}{h}\right)
$$


\subsection{Monte Carlo experiment}

\subsubsection{a} First, we want to compute the theoretically optimal bandwidth for $s=0$, $n=1000$, using the Epanechnikov kernel $(P=2)$, with the following Gaussian DGP:
$$x_i \sim 0.5 \N(-1.5,-1.5) + 0.5\N(1,1)$$

Filling in what we know so far we have : 
$$h_{AIMSE_s} = \left[ \frac{\vartheta_0(K)}{\vartheta_{2}(f) \cdot \mu_2(K)^2} \frac{1}{1000} \right]^{\frac{1}{5}}$$

So we need the second moment of K and the first moment of the second derivative of k squared. We can get two of these values from the table in Bruce Hanson's nonparametric notes. Giving us. 
$$h_{AIMSE_s} = \left[ \frac{\frac{3}{5}}{\vartheta_{2}(f) \cdot \frac{1}{5}^2} \frac{1}{1000} \right]^{\frac{1}{5}}$$

The second derivative of the normal density $\varphi$ with mean $\mu$ variance $\sigma^2$ is 

$$\varphi''_{\mu, \sigma^2}(x) = \frac{1}{\sqrt{2 \pi \sigma^2 }}e^{\frac{-(x-\mu)^2}{2\sigma^2}} \left[ \left( \frac{(x - \mu)}{\sigma^2} \right)^2 - \frac{1}{\sigma^2} \right]
$$

now useing the linearity of integrals we can find $\vartheta_{2}(f)$
$$ \vartheta_{2}(f) = \int_{-\infty}^{\infty} [0.5 \varphi''_{1,1}(x) + 0.5 \varphi''_{-1.5, 1.5}(x)]^2dx \approx 0.03883397
$$

Where the approximation comes from R 

Finally, pluging this in gives the theoretically optimal bandwidth is: 

$$h* = 0.8267532
$$

\subsubsection{b}

Below Is the table of $\widehat{IMSE}^{lI}$ results and $\widehat{IMSE^{LO}}$ results by bandwidth $h$ 

\begin{center}
	% latex table generated in R 3.3.2 by xtable 1.8-2 package
% Tue Oct 09 03:18:06 2018
\begin{tabular}{lll}
  \hline
imse\_li & imse\_lo & h \\ 
  \hline
0.000172 & 0.000172 & 0.5 \\ 
  0.000141 & 0.000142 & 0.6 \\ 
  0.000119 & 0.000122 & 0.7 \\ 
  0.000115 & 0.000117 & 0.8 \\ 
  0.000113 & 0.000116 & 0.9 \\ 
  0.00012 & 0.000124 & 1 \\ 
  0.000134 & 0.000138 & 1.1 \\ 
  0.000156 & 0.000159 & 1.2 \\ 
  0.000185 & 0.000189 & 1.3 \\ 
  0.000222 & 0.000226 & 1.4 \\ 
  0.00026 & 0.000264 & 1.5 \\ 
   \hline
\end{tabular}

\end{center}

\subsubsection{c}
Intuitively the difference between the two estimators, LI and LO, is that the LI includes the extra zero term in the sum since we include $x_i - x_i$. As the size of the sample increases this contribution to the overal average will go to zero. Meaning that the LI IMSE will also converge to the correct estimate. s

%------------------------------------------------
% end doc
%------------------------------------------------
\end{document}


