%-----------------------------------------------------------------------------------
%	PACKAGES AND OTHER DOCUMENT CONFIGURATIONS
%----------------------------------------------------------------------------------

\documentclass[11pt]{article}

\usepackage[top=2cm, bottom=3cm, left=2cm, right=2cm]{geometry}
\setlength{\parskip}{1em}
\setlength{\parindent}{4em}
\linespread{1.25}

\newcommand{\Var}{\mathrm{Var}}

\newcommand{\Cov}{\mathrm{Cov}}

\newcommand{\plim}{\rightarrow_{p}}

\usepackage{apacite}

\usepackage{amsmath, amsfonts}
\usepackage{graphicx}
\usepackage{pdfpages}
\usepackage{bm}
\usepackage{listings}
\usepackage{multirow,array}
\usepackage{enumerate}
\usepackage{bbm}
\usepackage{subfig}


\usepackage[latin1]{inputenc}

\usepackage{amssymb}

\usepackage{mathrsfs}
\usepackage{float}
\usepackage{booktabs}
\usepackage{color}
\usepackage{rotating}
\usepackage{amsthm}
\usepackage{multirow,array}
\usepackage{caption}
\usepackage{url}



\DeclareMathOperator*{\argmax}{arg\,max}
\DeclareMathOperator*{\argmin}{arg\,min}



% Expectation symbol
\newcommand{\E}{\mathrm{E}}
\newcommand{\V}{\mathrm{V}}
\newcommand{\N}{\mathcal{N}}
\newcommand{\R}{\mathbb{R}} 

%----------------------------------------------------------------------------------
%	TITLE AND AUTHOR(S)
%----------------------------------------------------------------------------------

\title{Welfare or Well-Unfair? Incorporating Heterogeneous Ability to Pay Into Normative Analysis} % The article title

\author{Nathan Mather} % The article author(s) 

\date{\today} % An optional date to appear under the author(s)

\renewcommand{\contentsname}{Table of Contents}


%----------------------------------------------------------------------------------
\begin{document}
	
	
%------------------------------------------------------------------------------
%	TABLE OF CONTENTS
%------------------------------------------------------------------------------
\maketitle % Print the title/author/date block

\setcounter{tocdepth}{3} % Set the depth of the table of contents to show sections and subsections only


\newpage

%------------------------------------------------------------------------------
% Introduction 
%------------------------------------------------------------------------------

Allocating scarce resources is, at its core, what many economic policy discussions are about. While we can look for pareto improvements or identify causal links, at the end of the day, a policy decision often needs to be made. To inform these decisions economists must rely on normative economic evaluations. Normative economics is inherently subjective, but economists can provide an analytically consistent approach to considering costs and benefits. This approach typically rests on the concept of willingness to pay, and willingness to pay is typically translated directly into a welfare measure denominated in dollars (economic surplus). This measure gives no consideration to the income or means of the consumer. My paper idea is to consider a consumer's ability to pay in addition to their willingness to pay when estimating welfare. This conversion will rely on an equity parameter corresponding to a given policymaker's normative view of welfare for a given willingness to pay and income. Using micro BLP methods we can estimate an income reweighted welfare measure for a policy and a list of equity parameters to determine how much a policy maker would need to care, or not care, about equity in order to choose one policy over another.


The core idea of what I am hoping to accomplish with this new method can be illustrated with a simple example. Consider allocating a banana to either Jeff Bezos, or me. An economist would most likely try to answer this question with something like an auction. In essence, how much are the two of us willing to pay? Jeff Bezos, a multi billionaire, may be willing to pay a lot, say \$20, while I, a grad student on a stipend, would pay maybe \$1. Do we really believe that Jeff Bezos gets 20 times more welfare from the banana than me? Maybe, but maybe not. 

This approach could be justified because knowing our willingness to pay is informative. I think it is apparent that willingness to pay is not the entire story, but in the case of a banana, using willingness to pay and leaving the equity consideration to the reader may be acceptable. The normative tradeoff that needs consideration is fairly clear. But, what about in a more complicated policy setting? What about when considering a merger policy that impacts prices of goods along a spectrum of quality parameters heterogeneously? What if these goods are consumed by agents with heterogeneous incomes? Can we expect the reader to accurately assess the equity concerns then? I would argue no. This is placing too high of a burden on the reader. Instead, we need a metric to capture the policymaker's preferences and reduce the dimensionality of the normative considerations. 

%----------------------------------------------------------
% Method 
%----------------------------------------------------------


This idea can be achieved in a basic binary choice demand setting with little effort. This first example should emphasize that, at its core, what I am proposing is not complicated. To start, consider the following question:  ``For which X would the following make roughly the same difference? One thousand dollars to a family with an income like yours, or X dollars to a family with half your family's income?" The answer to this question informs the normative ``equity parameter" which weights a given consumer's willingness to pay based on their income. This is subjective, so the best option is to choose a range of responses and provide a menu of calculations to the reader. 

Let $CS =$ Consumer Surplus, $D(i)=$ demand for consumer $i$, $P =$ Price, $K =$ Number of consumers, 
$\bar{M} =$ mean income, $I_i =$ Income, $\phi =$ Equity parameter, $W = $ Welfare   

Discrete consumer surplus could be calculated like so: 

$$  CS = \sum_{Q = 1}^{K} (D(i) - P) $$

But, from the answer to our above normative question we can derive a ``willingess to pay" to ``welfare" weighting function. This function maps the answer to the income question to the parameter $\phi$. Note that this function is in utils and so is simply normalized to 1 at the mean income. 

$$ N(i) = \frac{\bar{M}}{I_i^\phi} $$

Now we can derive a truly normative metric for welfare

$$  W = \sum_{Q = 1}^{K} (D(i) - P) \cdot \frac{\bar{M}}{I_i^\phi}  $$

While I stand by the idea that this is not a complicated extension in theory, actually estimating this parameter is another story. To calculate this exactly we would need to know every person's willingness to pay and their matched incomes. While this seems like a challenge we do not literally need this information for every consumer to get an estimate. Just like with willingness to pay, we can estimate our welfare weights $N(i)$. The general Idea is laid out below. 

$\mu_{ij} = U_{ij} + \epsilon_{ij}$

 $\mu_{i0} = \epsilon_{i0}$
 
 $\epsilon_{ij}$ i.i.d. $\sim$ type-1 extreme value distribution 
 
 $U_{ij}$ is linear in price with coefficient $\alpha_i$
 

Now we can get compensating variation as 

$$ CV_i = \frac{V_i^1 - V_i^0}{\alpha_i} N(i) $$ 

where 
$$ V_i = ln(1 + \sum_{j = 1}^{J} e^{U_{ij}})$$ 

Now we can take the expectation of this over observable demographics (including income which impacts our weights) and unobservable taste shocks. 


This idea certainly has some setbacks. That being said I believe the issues are minor and advantages large. One of the most apparent issues with this approach is that the welfare metric is no longer in dollars. In fact it is now only unique up to a linear transformation. In the equations above I normalized the welfare metric using mean income. On it's face this seems like we are giving up a lot. However, dollar quantities of surplus are a bit vague and hard to consider in isolation anyway. What is a million dollars of surplus? How does this compare to a million dollars in the hands of the government or an individual? The most effective way to understand this quantity is to compare it to the surplus of an alternate scenario or market. For comparisons like these, a metric in utils works just as well as one in dollars. 

A potentially complicated issue is extending this concept beyond consumer surplus. If we are measuring producer surplus as well, how do we weight that in utils? Since businesses are owned by people ideally I would propose we weight producer surplus by the expected income weights of the owners of the firm. This, however, may be difficult to ascertain. One way to address this issue is to add an additional normative parameter for weighting of consumer and producer surplus. This way the reader can consider the range of possible answer from zero weight on producers to all weight on producers. This does however leave us with two normative parameters and increase the dimensionality of our menu of potential answers. This may be acceptable but the farther we move away from a definitive and clear answer the less useful the analysis may be. Perhaps including a reasonable estimate for the firm owner's income as well as a range around that is a reasonable option. 

Another issue that could be raised is the implicit assumption that the equity parameter is quadratic in income. It could easily be different, but we do need some type of parametric assumption. In relative terms this is actually a strength and not a weakness. Economic surplus also assumes a functional form of $ n(i) = 1$ and is in fact a special case of my metric.
 
Finally, while it is apparent that willingness to pay may mean different welfare for different people, income is not the only dimension that could impact this. Perhaps consumption or wealth would be a better indicator of ability to pay. There may even be aspects like health that could impact how much welfare is derived from a given willingness to pay. This idea is flexible enough to consider these variables as well wherever they are available. While it may be imperfect to consider only one, it is still an improvement on considering none. 

In order to develop the method and idea into a paper I plan on doing the following. First, I will chose a micro BLP paper with income data that I can replicate. It might make sense to use the BLP 2004 paper and consider the introduction of the minivan like Petrin \cite{Berry2004} \cite{Petrin2002}. However a more recent policy oriented paper may be better. I am still looking around for a good candidate. After replication I will try to implement my idea. In addition I would try to find other ways to more clearly visualize the equity trade off. Professor Fan suggested graphing the entire distribution of willingness to pay and income before compressing it with parametric weight. I really like this idea and would like to spend more time thinking about data visualization of this type. I would develop this method and an visualization techniques into an R package. Finally,  I would implement this on a handful of previous studies, the more the better, and see to what extent these normative parameters alter the picture of efficient policy. 


\section{sources}


\bibliographystyle{apacite}
\bibliography{References}
%------------------------------------------------
% end doc
%------------------------------------------------
\end{document}