%-----------------------------------------------------------------------------------
%	PACKAGES AND OTHER DOCUMENT CONFIGURATIONS
%----------------------------------------------------------------------------------

\documentclass{beamer}


\newcommand{\Var}{\mathrm{Var}}

\newcommand{\Cov}{\mathrm{Cov}}

\newcommand{\plim}{\rightarrow_{p}}

\usepackage{apacite}

\usepackage{amsmath, amsfonts}
\usepackage{graphicx}
\usepackage{pdfpages}
\usepackage{bm}
\usepackage{listings}
\usepackage{multirow,array}
\usepackage{enumerate}
\usepackage{bbm}
%\usepackage{subfig}
\usepackage{bbm}
\usepackage{multirow}
\usepackage[space]{grffile}
\usepackage{subcaption}


\usepackage{amssymb}

\usepackage{mathrsfs}
\usepackage{float}
\usepackage{booktabs}
\usepackage{color}
\usepackage{rotating}
\usepackage{amsthm}
\usepackage{multirow,array}
\usepackage{caption}
\usepackage{url}



\DeclareMathOperator*{\argmax}{arg\,max}
\DeclareMathOperator*{\argmin}{arg\,min}



% Expectation symbol
\newcommand{\E}{\mathrm{E}}
\newcommand{\V}{\mathrm{V}}
\newcommand{\N}{\mathcal{N}}
\newcommand{\R}{\mathbb{R}} 

\setbeamertemplate{navigation symbols}{}


%-----------------------------
% title stuff 
%-------------------------

\usepackage[utf8]{inputenc}



%Information to be included in the title page:
\title{Welfare or Well-Unfair: Incorporating Heterogeneous Income Into Normative Analysis}
\author{Nathan Mather}
\institute{University of Michigan}
\date{2019}





%------------------------------------------------------------------------------
%	begin doc
%------------------------------------------------------------------------------

\begin{document}
	
	\frame{\titlepage}
	
\section{Introduction}

%------------------------------------------------------------------------------
%	Outline Of Presentation 
%------------------------------------------------------------------------------
\begin{frame}
	\frametitle{Outline}
	\begin{enumerate}
		\setlength{\itemsep}{8mm}
		 \large
		 \item Show (or remind) us why typical welfare estimations are a flawed tool for normative economics 
		\item Outline the general idea of what I hope to do 
		\item Show the shell of a method 
	\end{enumerate}
\end{frame}

%------------------------------------------------------------------------------
% Motivating example   
%------------------------------------------------------------------------------


% fram one 
\begin{frame}
\frametitle{Motivating Example}

\begin{figure}
	\centering
\begin{subfigure}[b]{.3\linewidth}
	\includegraphics[width=\linewidth]{banana.jpeg}
	\caption{A Banana}\label{fig:banana}
\end{subfigure}


\begin{subfigure}[b]{.3\linewidth}
	\includegraphics[width=\linewidth]{Bezos_pic_1_square.jpg}
	\caption{Jeff Bezos}\label{fig:Bezos}
\end{subfigure}
\begin{subfigure}[b]{.3\linewidth}
	\includegraphics[width=\linewidth]{nate_square.jpg}
	\caption{Me}\label{fig:Nate}
\end{subfigure}
\end{figure}

\end{frame}

% frame 2 
\begin{frame}
\frametitle{Motivating Example}

	\begin{itemize}
			\setlength{\itemsep}{5mm}
		\large
	\item Who should get the banana? \\\\

	\item Economists often use ``How much are you willing to pay?" 
	
		\item Jeff -``I mean it’s one banana Nathan. What could it cost, \$10? look I'd pay \$100"\footnote[frame]{Not an actual quote} 
		
		\item Nate - ``I missed lunch and am really hungry, I would pay \$5"\footnote[frame]{Also not an actual quote. \$5 for a banana?} 
		
		\item Who get's more welfare from the banana?
			
	\end{itemize}


\end{frame}

% frame 3
\begin{frame}
\frametitle{Motivating Example}

\begin{itemize}
	\setlength{\itemsep}{8mm}
	\large
	\item 
\end{itemize}


\end{frame}


%------------------------------------------------
%  Thank you frame 
%------------------------------------------------
\begin{frame}

\frametitle{The End }

\begin{center}
	
\begin{Huge}
\textbf{Thank You}
\end{Huge}
\end{center}
\end{frame}

%------------------------------------------------
% end doc
%------------------------------------------------
\end{document}

