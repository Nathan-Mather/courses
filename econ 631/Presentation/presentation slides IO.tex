%-----------------------------------------------------------------------------------
%	PACKAGES AND OTHER DOCUMENT CONFIGURATIONS
%----------------------------------------------------------------------------------

\documentclass{beamer}


\newcommand{\Var}{\mathrm{Var}}

\newcommand{\Cov}{\mathrm{Cov}}

\newcommand{\plim}{\rightarrow_{p}}

\usepackage{apacite}

\usepackage{amsmath, amsfonts}
\usepackage{graphicx}
\usepackage{pdfpages}
\usepackage{bm}
\usepackage{listings}
\usepackage{multirow,array}
\usepackage{enumerate}
\usepackage{bbm}
%\usepackage{subfig}
\usepackage{bbm}
\usepackage{multirow}
\usepackage[space]{grffile}
\usepackage{subcaption}


\usepackage{amssymb}

\usepackage{mathrsfs}
\usepackage{float}
\usepackage{booktabs}
\usepackage{color}
\usepackage{rotating}
\usepackage{amsthm}
\usepackage{multirow,array}
\usepackage{caption}
\usepackage{url}



\DeclareMathOperator*{\argmax}{arg\,max}
\DeclareMathOperator*{\argmin}{arg\,min}



% Expectation symbol
\newcommand{\E}{\mathrm{E}}
\newcommand{\V}{\mathrm{V}}
\newcommand{\N}{\mathcal{N}}
\newcommand{\R}{\mathbb{R}} 

\setbeamertemplate{navigation symbols}{}


%-----------------------------
% title stuff 
%-------------------------

\usepackage[utf8]{inputenc}



%Information to be included in the title page:
\title{Welfare or Well-Unfair: Incorporating Heterogeneous Income Into Normative Analysis}
\author{Nathan Mather}
\institute{University of Michigan}
\date{2019}





%------------------------------------------------------------------------------
%	begin doc
%------------------------------------------------------------------------------

\begin{document}
	
	\frame{\titlepage}
	
\section{Introduction}

%------------------------------------------------------------------------------
%	Outline Of Presentation 
%------------------------------------------------------------------------------
\begin{frame}
	\frametitle{Outline}
	\begin{enumerate}
		\setlength{\itemsep}{8mm}
		 \large
		 \item Show (or remind) us why typical welfare estimations are a flawed tool for normative economics 
		\item Outline the general idea of what I hope to do 
		\item Show the shell of a method 
	\end{enumerate}
\end{frame}

%------------------------------------------------------------------------------
% Motivating example   
%------------------------------------------------------------------------------


% fram one 
\begin{frame}
\frametitle{Motivating Example}

\begin{figure}
	\centering
\begin{subfigure}[b]{.3\linewidth}
	\includegraphics[width=\linewidth]{banana.jpeg}
	\caption{A Banana}\label{fig:banana}
\end{subfigure}


\begin{subfigure}[b]{.3\linewidth}
	\includegraphics[width=\linewidth]{Bezos_pic_1_square.jpg}
	\caption{Jeff Bezos}\label{fig:Bezos}
\end{subfigure}
\begin{subfigure}[b]{.3\linewidth}
	\includegraphics[width=\linewidth]{nate_square.jpg}
	\caption{Me}\label{fig:Nate}
\end{subfigure}
\end{figure}

\end{frame}

% frame 2 
\begin{frame}
\frametitle{Motivating Example}

\begin{itemize}
	\setlength{\itemsep}{5mm}
	\large
	\item Who should get the banana? 
	
	\item Economists often use ``How much are you willing to pay?" 
	
	\item Jeff -``I mean it’s one banana Nathan. What could it cost, \$10? look I'd pay \$100"\footnote[frame]{Not an actual quote} 
	
	\item Nate - ``I missed lunch and am really hungry, I would pay \$5"\footnote[frame]{Also not an actual quote. \$5 for a banana?} 
	
	\item Who get's more welfare from the banana?
\end{itemize}

\end{frame}



% frame 
\begin{frame}
\frametitle{Justification For Willingness to Pay}

\begin{itemize}
	\setlength{\itemsep}{5mm}
	\large 
	\item We are Maximizing the size of the “pie” and we can redistribute later 
	\begin{itemize}
			\setlength{\itemsep}{5mm}
		\large
		\item While this may be true in some sense the pie is typically not redistributed
	\end{itemize}
	\item  We are getting a sense of the "cost" of a policy and then the reader can decide which is better based on equity concerns 
	\begin{itemize}
			\setlength{\itemsep}{5mm}
		\large
		\item The equity trade off is pretty clear in the banana example 
		\item What about more complicated policies impacting various groups? 
	\end{itemize}
\end{itemize}


\end{frame}


% frame 
\begin{frame}
\frametitle{More Complicated Examples}

\begin{itemize}
	\setlength{\itemsep}{5mm}
	\large
	\item Deciding between a tax on rice and caviar 
	\item Allowing a merger that raises the price of low quality goods but lowers price and cost of high quality goods 
	\item Deciding on health-care mandates, subsidies or restrictions 
	\item Replacing old technology with new
	\item In these examples the normative equity trade-offs are harder to wrap our heads around
	
\end{itemize}

\end{frame}


% frame 
\begin{frame}
\frametitle{Main Goal}

\begin{itemize}
	\setlength{\itemsep}{5mm}
	
\large
		
	\item Reduce the number of comparisons we leave to the reader
	\item i.e. reduce the dimensionality of the problem 
	\item Make these policy trade-offs more comparable to the banana problem 
	\item Create Normative parameter to capture the Equity Trade-off

\end{itemize}

\end{frame}

% frame 
\begin{frame}
\frametitle{Basic Example}

Let $CS =$ Consumer Surplus, $D(i)=$ demand for consumer $i$, $P =$ Price, $K =$ Number of consumers, $\bar{M} =$ mean income, $I_i =$ Income, $W = $ Welfare   

 Discrete consumer surplus could be calculated like so: 
 
 $$  CS = \sum_{Q = 1}^{K} (D(i) - P) $$
 
 But, from the answer to our above question we can derive a willingess to pay to ``welfare" weights 
 
 $$ N(i) = \frac{\bar{M}}{I_i^2} $$
 
 Now we can derive a truly normative metric for welfare in the market 
 
  $$  W = \sum_{Q = 1}^{K} (D(i) - P) \cdot \frac{\bar{M}}{I_i^2}  $$

 

\end{frame}


% frame 
\begin{frame}
\frametitle{BLP Example}




\end{frame}




%frame 
\begin{frame}
\frametitle{Informing a Normative Choice}

\begin{itemize}
	
	\item Use something like the following:
	\begin{itemize}
		
		
		\item For which X would the following make roughly the same difference? One thousand dollars to a family with an income like yours, or $X$ dollars to a family with half your family's income? 
	\end{itemize}
	
	\item Provides us with a way to translate surplus from a given individual into a subjective welfare measure incorporating income
	\item Outcome is normative (as it shouold be) 
	\item We can provide welfare analysis for a menu of different responses and report them back
\end{itemize}	


\end{frame}



%------------------------------------------------
%  Thank you frame 
%------------------------------------------------
\begin{frame}

\frametitle{The End }

\begin{center}
	
\begin{Huge}
\textbf{Thank You}
\end{Huge}
\end{center}
\end{frame}

%------------------------------------------------
% end doc
%------------------------------------------------
\end{document}

